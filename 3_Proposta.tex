% !TeX spellcheck = <none>
%!TEX root = Projeto.tex
\chapter{Proposta}

%!TeX spellcheck = pt_BR
%!TEX root = Projeto.tex

\section{API de ocultação do DOM fundamentada em Shadow DOM}
\tbc
%!TeX spellcheck = pt_BR
%!TEX root = Projeto.tex

\section{Casos de teste de exposição aos riscos de vazamento de informação no navegador}
\tbc
%!TeX spellcheck = pt_BR
%!TEX root = Projeto.tex

\section{Apresentação do site acadêmico ``jstest.me''}
\tbc

\section{Roteiro para a elaboração da proposta}

\begin{alineas}
	\item \textbf{Mapeamento de cenários.} Partindo dos requisitos não funcionais estabelecidos, serão definidos cenários de testes capazes de evidenciar as condições necessárias para a não-conformidade com esses requisitos. Os cenários assim definidos delimitarão o escopo de ação da abordagem proposta por este trabalho.
	\item \textbf{Especificação da proposta.} A abordagem derivada dos cenários de testes será formalmente especificada, em termos das interfaces e funcionalidades esperadas, visando orientar o desenvolvimento de uma implementação de teste e provas de conceito.
	\item \textbf{Implementação do mecanismo de ocultação.} A materialização da abordagem proposta será efetuada pelo desenvolvimento dos scripts necessários para a implementação das interfaces esperadas, em conformidade com os requisitos funcionais esperados.
	\item \textbf{Validação.} Os cenários de testes serão implementados na forma de provas de conceito para validação da proposta e do mecanismo implementado. O objetivo é conhecer a aderência do mecanismo em relação aos requisitos.
\end{alineas}