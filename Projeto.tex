%!TeX program = xelatex
%
% O MODELO DESTE DOCUMENTO É abntex2-modelo-trabalho-academico.tex
% HÁ MAIS SEÇÕES NO MODELO DO QUE NESTE ARQUIVO AQUI.
% É PARA CONSTRUIR AOS POUCOS.
%
%
\documentclass[
	% -- opções da classe memoir --
	12pt,				% tamanho da fonte
	openright,			% capítulos começam em pág ímpar (insere página vazia caso preciso)
	twoside,			% para impressão em recto e verso. Oposto a oneside
	a4paper,			% tamanho do papel. 
	% -- opções da classe abntex2 --
	%chapter=TITLE,		% títulos de capítulos convertidos em letras maiúsculas
	%section=TITLE,		% títulos de seções convertidos em letras maiúsculas
	%subsection=TITLE,	% títulos de subseções convertidos em letras maiúsculas
	%subsubsection=TITLE,% títulos de subsubseções convertidos em letras maiúsculas
	% -- opções do pacote babel --
	english,			% idioma adicional para hifenização
	brazil				% o último idioma é o principal do documento
	]{abntex2}

\usepackage{_estilos}

% ====================================================================================================================================
% ---
% Informações de dados para CAPA e FOLHA DE ROSTO
% ---
\titulo{Método para encapsulamento da informação em aplicações de HTML e Javascript}
\autor{Humberto Borgas Bulhões}
\local{São Paulo}
\data{2017}
\mes{Fevereiro}
\orientador{Marcelo Novaes de Rezende}
\instituicao{\mbox{Instituto} de Pesquisas Tecnológicas do \mbox{Estado} de São Paulo}
\tipotrabalho{Dissertação de Mestrado}
\curso{Engenharia da Computação: Engenharia de Software}
\concentracao{Área de Concentração: Engenharia de \mbox{Software}}
\tipotrabalho{Dissertação de Mestrado}
% O preambulo deve conter o tipo do trabalho, o objetivo, 
% o nome da instituição e a área de concentração 
\preambulo{Exame de Qualificação apresentado ao
	Instituto de Pesquisas Tecnológicas do
	Estado de São Paulo – IPT, como parte
	dos requisitos para a obtenção do título
	de mestre em Engenharia de
	Computação.}
% ---


% ====================================================================================================================================
% ---
% compila o indice
% ---
%\makeindex
% ---

% ====================================================================================================================================
% ----
% Início do documento
% ----
\begin{document}

\chapterstyle{iptabntex2}			% Estilo de capitulo adotado pelo IPT

% Retira espaço extra obsoleto entre as frases.
\frenchspacing

% ----------------------------------------------------------
% ELEMENTOS PRÉ-TEXTUAIS
% ----------------------------------------------------------
\pretextual
%!TEX root = Projeto.tex
\imprimircapa
\imprimirfolhadeaprovacao
\imprimirfolhaderosto

% ---
% Inserir a ficha bibliografica
% ---

% Isto é um exemplo de Ficha Catalográfica, ou ``Dados internacionais de
% catalogação-na-publicação''. Você pode utilizar este modelo como referência.
% Porém, provavelmente a biblioteca da sua universidade lhe fornecerá um PDF
% com a ficha catalográfica definitiva após a defesa do trabalho. Quando estiver
% com o documento, salve-o como PDF no diretório do seu projeto e substitua todo
% o conteúdo de implementação deste arquivo pelo comando abaixo:
%
% \begin{fichacatalografica}
%     \includepdf{fig_ficha_catalografica.pdf}
% \end{fichacatalografica}
\begin{fichacatalografica}
	\vspace*{\fill}					% Posição vertical
	\hrule							% Linha horizontal
	\begin{center}					% Minipage Centralizado
		\begin{minipage}[c]{12.5cm}		% Largura
			\setlength{\parskip}{1em}

			\imprimirautor
		
			\hspace{0.5cm} \imprimirtitulo~/~\imprimirautor. \imprimirlocal, \imprimirdata.

			\hspace{0.5cm} \pageref{LastPage} p. : il. (algumas color.); 30 cm.
		
			\hspace{0.5cm} \imprimirorientadorRotulo~\imprimirorientador
		
			\hspace{0.5cm} \parbox[t]{\textwidth}{\imprimirtipotrabalho~--~\imprimirinstituicao, \imprimirdata.}
		
			\hspace{0.5cm}
		%		1. Palavra-chave1.
		%		2. Palavra-chave2.
		%		I. Orientador.
		%		II. Universidade xxx.
		%		III. Faculdade de xxx.
		%		IV. Título\\
		
			\hspace{8.75cm} CDU XX:XXX:XXX.X\\
	
		\end{minipage}
	\end{center}
	\hrule
\end{fichacatalografica}
% ---



% ---
% Dedicatória
% ---
%\begin{dedicatoria}
%\vspace*{\fill}
%\OnehalfSpacing
%%\centering
%%\noindent
%Dedico este trabalho...
%
%
%\vspace*{\fill}
%\end{dedicatoria}
% ---

% ---
% Agradecimentos
% ---
%\begin{agradecimentos}
%\vspace*{\fill}
%\OnehalfSpacing
%Gostaria de agradecer...
%
%
%\vspace*{\fill}
%\end{agradecimentos}
% ---

% ---
% Epígrafe
% ---
%\begin{epigrafe}
%    \vspace*{\fill}
%	\begin{flushright}
%		\textit{``Não vos amoldeis às estruturas deste mundo, \\
%		mas transformai-vos pela renovação da mente, \\
%		a fim de distinguir qual é a vontade de Deus: \\
%		o que é bom, o que Lhe é agradável, o que é perfeito.\\
%		(Bíblia Sagrada, Romanos 12, 2)}
%	\end{flushright}
%\end{epigrafe}
% ---


%!TEX root = Projeto.tex
\newpage
\begin{resumo}
\normalsize

% Breve contextualização do problema
Os programas navegadores oferecem amplos recursos para a execução de aplicações voltadas para a web. Suas funcionalidades, porém, podem expor a informação dos usuários a riscos de vazamento e adulteração. Ainda que os navegadores venham sendo melhorados para que consigam detectar e mitigar alguns desses riscos, esse esforço é marcado por concessões às capacidades esperadas pelas aplicações web, fazendo com que a segurança da informação no navegador seja um campo de conhecimento com iniciativas e padrões consideradas incoerentes entre si. Por causa disso, desenvolvedores de aplicações nem sempre podem prever o grau de exposição dos dados de seus usuários.
% Dica do que pode ser feito para minimizar o problema
Seria importante que fosse possível determinar, de modo programável e imperativo, que esses dados ficassem fora do alcance de participantes não confiáveis, particularmente \scripts.
% Proposta do trabalho
Para essa finalidade, este trabalho propõe uma abordagem que proporcione ao desenvolvedor uma barreira de proteção incorporada às aplicações. Nesta proposta, o desenvolvedor tem controle direto sobre a exposição das informações que considerar sensíveis, em contraste com o controle indireto, declarativo e tolerante a vazamento de informação empregado atualmente nos navegadores.
% Como será validado
A validação dessa proposta ocorrerá por meio de um protótipo de sistema submetido a situações de risco de vazamento da informação em páginas web, em correspondência com ocorrências documentadas pela literatura. O protótipo será preparado para que a abordagem proposta, sob a forma de um \script baseado em APIs padronizadas de HTML e Javascript, seja colocada à prova no seu propósito de neutralizar tais situações de risco.


\vspace{\onelineskip}

\noindent
\textbf{Palavras-chave:} \imprimirpalavraschave
\end{resumo}

% resumo em inglês
%\begin{resumo}[Abstract]
%%\begin{otherlanguage*}{english}
%Resumo da dissertação em inglês.
%
%
%\vspace{\onelineskip}
%
%\noindent
%\textbf{Keywords:} \imprimirkeyword   
%%\end{otherlanguage*}
%\end{resumo}

% ---
% inserir lista de ilustrações
% ---
{\SingleSpacing
\pdfbookmark[0]{\listfigurename}{lof}
\listoffigures*
\cleardoublepage
}
% ---

% ---
% inserir lista de trechos de código fonte
% ---
{\SingleSpacing
	\pdfbookmark[0]{\lstlistlistingname}{lol}
	\lstlistoflistings
	\cleardoublepage
}
% ---

% ---
% inserir lista de quadros
% ---
%\pdfbookmark[0]{\listofquadrosname}{loq}
%\listofquadros*
%\cleardoublepage
% ---

% ---
% inserir lista de tabelas
% ---
%{\SingleSpacing
%\pdfbookmark[0]{\listtablename}{lot}
%\listoftables*
%\cleardoublepage
%}
% ---

% ---
% inserir lista de abreviaturas e siglas
% ---
%
\begin{siglas}
  \item[CORS] Cross-Origin Resource Sharing
  \item[CSRF] Cross-Site Request Forgery
  \item[CSP] Content Security Policy
  \item[DAC] Discretionary Access Control
  \item[DOM] Document Object Model
  \item[IFC] Information Flow Control
  \item[JSON] JavaScript Object Notation
  \item[MAC] Mandatory Access Control
  \item[SOP] Same Origin Policy
  \item[SRI] Subresource Integrity
  \item[XSS] Cross-Site Scripting
\end{siglas}

% ---

% ---
% inserir lista de símbolos
% ---
%\begin{simbolos}
%  \item[$ \Gamma $] Letra grega Gama
%  \item[$ \Lambda $] Lambda
%  \item[$ \zeta $] Letra grega minúscula zeta
%  \item[$ \in $] Pertence
%\end{simbolos}
% ---

% ---
% inserir o sumario
% ---
{\SingleSpacing
	\cleardoublepage
	\pdfbookmark[0]{\contentsname}{toc}
	\tableofcontents*
	\cleardoublepage
}
% ---


% ----------------------------------------------------------
% ELEMENTOS TEXTUAIS
% ----------------------------------------------------------
\textual
\OnehalfSpacing

%!TEX root = Projeto.tex
\chapter{Introdução}

 %!TEX root = Projeto.tex
\section{Motivação}

%ESTÁ MUITO CONFUSO. MELHOR COMEÇAR DIRETO AO PONTO COMO O DEIAN. DEIXAR EVIDENTE A DISTÂNCIA ENTRE O ESTADO DA ARTE E O ESTADO DAS FERRAMENTAS, E POR QUE ISSO COMPLICA A VIDA DE DESENVOLVEDORES E USUÁRIOS. ESCREVER PENSANDO QUE O OBJETIVO DO TRABALHO É ""AVALIAR"" UMA ALTERNATIVA AO ESTADO DA ARTE, SEM EFETIVAMENTE DIZÊ-LO AQUI.

A originação e o transporte de dados pela web são processos sujeitos a diversas preocupações relacionadas à segurança da informação. Seja na fronteira do usuário com a internet (o chamado \textit{front-end}), na fronteira da internet com os sistemas subjacentes (\textit{back-end}), bem como na infraestrutura entre as duas pontas, informações estão potencialmente expostas a adulteração e ao acesso não autorizado. Muitos dos riscos são mitigados com medidas aplicáveis durante o desenvolvimento, a publicação e a manutenção dos recursos que compõem as aplicações da web, tornando-a um território relativamente seguro e próspero para usuários e desenvolvedores. Ainda assim, uma parte da tecnologia que dá suporte a esses recursos é fundamentalmente limitada no nível de segurança da informação que pode oferecer, especialmente quando é considerado o rico ambiente de execução proporcionado pelo software navegador. Nele, a confidencialidade da informação está sujeita a ataques -- vazamentos -- provocados por agentes incorporados tanto nas páginas da web e quanto no próprio navegador.

Agentes do primeiro caso tiram proveito de funcionalidades de HTML e Javascript para a composição de experiências de usuário em conteúdo misto, em que uma mesma interface (página da web) hospeda componentes publicados por diferentes provedores de serviços sem que estes tenham pré-estabelecido qualquer relação de confiança uns com os outros. Dado que esses componentes têm os mesmos privilégios e mesmo nível de acesso à página como um todo \cite[p. 2-3]{DeRyck2012}, caberia ao navegador aplicar diretivas de segurança que impedissem vazamentos de informação entre os componentes. Algumas dessas diretivas existem, mas sua aplicação é limitada \cite{Jang2010}.

O segundo caso se baseia na extensibilidade do navegador. Navegadores modernos oferecem recursos de expansão por software acessório denominado \textit{plugins} ou extensões. O software acessório tem nível de acesso mais elevado aos recursos do navegador do que aquele definido para componentes de páginas da web, permitindo que \textit{plugins} tomem controle de funcionalidades como gerenciamento de conexões de rede, eventos de navegação e o DOM. Ainda que dependam da permissão explícita do usuário no momento da incorporação ao navegador, os \textit{plugins} e extensões podem se comportar como agentes de vazamento de informação de forma sutil e não necessariamente proposital \cite{Heule2015}.

%(REFORMULAR DAQUI EM DIANTE)

Segurança da informação é um tópico importante na padronização das tecnologias que dão suporte à web. Protocolos de comunicação implementam criptografia, navegadores submetem seus módulos a restrições e limites de operação, e os desenvolvedores são desencorajados a expor informação sigilosa através das funcionalidades de programação. Mas a despeito de restrições impostas pelos navegadores para a minimização dos riscos à segurança da informação, o uso malicioso ou ingênuo de certas funcionalidades coloca em risco a integridade de aplicações \textit{mashup} e de extensões do navegador. A forma como os diversos componentes de \textit{front-end} são integrados abre vulnerabilidades:

\begin{alineas}
	\item \textit{scripts} de diferentes origens podem ser incorporados em uma mesma página, compartilhar informações e transmiti-las a outros domínios sem que o usuário esteja ciente;
	\item extensões de navegador podem ter acesso irrestrito ao DOM, monitorar as conexões à rede e modificar o funcionamento de funcionalidades sem o conhecimento do usuário;
	\item imagens, folhas de estilo e outros recursos podem ser incorporados como meio de efetuar ações e cometer vazamento de dados;
	\item trechos de \textit{scripts} podem ser incorporados sem o conhecimento do desenvolvedor e do usuário, permitindo a execução de código projetado para vazamento de dados.
\end{alineas}

Sustentando essas vulnerabilidades está o fato de que, nas condições correntes, é possível que um script efetue, indevidamente, acessos de leitura e escrita em informações que o usuário solicite (via requisições) ou informe (via formulários) para o navegador. Isso é derivado da forma inconsistente e parcial como a linguagem Javascript e as APIs do navegador tratam o isolamento entre scripts e dados de origens distintas. Este é um campo de pesquisas ativo \cite{Stefan2014}, \cite{Hedin2014}, \cite{Bichhawat2014}, \cite{Magazinius2014} e em busca por padronização \cite{W3C:WebAppSec}, mas que ainda não resultou em práticas, ferramentas e protocolos de amplo alcance pois dependem da adoção de políticas de segurança experimentais \cite{Hedin2014}, \cite{Bichhawat2014} e potencialmente degradantes de desempenho \cite[p. 14]{Stefan2014} por parte dos desenvolvedores de navegadores.

Buscando uma alternativa isenta de implementações experimentais, este trabalho propõe um método para o confinamento de informação através de recursos de HTML e Javascript conhecidos como \textit{Custom Elements}\footnote{Especificação de \textit{Custom Elements}: https://w3c.github.io/webcomponents/spec/custom/}. Com isso, seria possível encapsular informações em componentes interativos cuja estrutura é invisível tanto para o restante da página HTML como para ``plugins''. Este método é suportado pelas versões mais recentes dos navegadores. A conclusão deste trabalho se dará pela avaliação da efetividade do método como meio de ofuscamento de (1) informações incorporadas e exibidas na página e (2) do comportamento do usuário ao interagir com componentes HTML e Javascript desenvolvidos conforme o método proposto.


%!TEX root = Projeto.tex
\section{Objetivo}

%A MOTIVAÇÃO DEVE TER ESCLARECIDO QUE EXISTEM PROBLEMAS DE INFOSEC DIFÍCEIS DE SE CONTORNAR COM AS DIRETIVAS TRADICIONAIS. NESTA SEÇÃO DEVERÁ FICAR CLARO QUE UMA ABORDAGEM DIFERENTE ""PODE"" SER AVALIADA, E QUE É OBJETIVO DESTE TRABALHO FAZER ESSA AVALIAÇÃO.

%Para formular um Objetivo realista, é preciso ter uma hipótese (hipo-tese).
%Hipótese: uma afirmação que deverá ser testada (parece razoável, baseando-me no que li, ou no que observei, que é possível fazer X...). É preciso justificar a hipótese.

Buscando alternativas de desenvolvimento defensivo que não se baseiem em implementações experimentais, este trabalho investiga um método para o confinamento de informação através de recursos padronizados de HTML e Javascript. Interfaces de programação (APIs -- \textit{application programming interface}) suportadas (ou em vias de adoção) pelos navegadores modernos, como o \textit{shadow DOM} \cite{W3C:ShadowDOM} e o elemento \textit{<iframe>} \cite{MDN:iframes}, dispõem de recursos que possibilitariam o encapsulamento de informações em componentes de estrutura opaca, seja para \textit{scripts} incorporados na página HTML como para \textit{plugins} do navegador. Hipoteticamente, é possível propor um método que, combinando tais recursos de APIs, ofereça ao desenvolvedor uma ferramenta para a composição de páginas imunes a ataques de vazamento de informação. Implementando o método e colocando-o à prova por meio de testes que simulam esse tipo de ataque, o objetivo deste trabalho é avaliar a efetividade do método como meio de ofuscação (1) das informações incorporadas e exibidas na página e (2) do comportamento do usuário ao interagir com componentes HTML e Javascript desenvolvidos conforme o método proposto.

%!TEX root = Projeto.tex
\section{Resultados esperados e contribuições}

%A SEÇÃO "OBJETIVO" ESTABELECEU QUE O TRABALHO VAI TRATAR DA PROTEÇÃO (OU NÃO) QUE UMA TECNOLOGIA COMO "SHADOW DOM" OFERECE AOS USUÁRIOS DE PÁGINAS WEB. NESTA SEÇÃO: QUE RESULTADOS SÃO ESPERADOS DESSE ESTUDO [INCLUIR: REVISÃO/LANDSCAPE (SIM) DO ESTADO DA ARTE E DAS VULNERABILIDADES DOS NAVEGADORES] [INCLUIR: MAPEAMENTO DOS ASPECTOS QUALITATIVOS DE UMA SOLUÇÃO DE INFOSEC PARA JAVASCRIPT EM NAVEGADOR]? COMO ESSES RESULTADOS SÃO IMPORTANTES PARA O AVANÇO NO ESTADO DA ARTE? COMO OS ARTEFATOS DERIVADOS DO TRABALHO PODEM CONTRIBUIR PARA A SEGURANÇA DA INFORMAÇÃO EM PÁGINAS DA WEB?

Este trabalho contribui com a investigação do potencial de inviolabilidade da informação oferecido por tecnologias de ampla disponibilidade, como \textit{shadow DOM} \cite{W3C:ShadowDOM} e \textit{iframes}, em relação a uma abordagem de referência baseada em IFC (IFC -- \textit{information flow control}). O referencial é relevante pois IFC, que fundamentalmente redefine o fluxo de informação em Javascript, ofereceria o nível mais alto de segurança se fosse integrada por padrão às APIs dos navegadores. Aos desenvolvedores e usuários é oportuno, portanto, que sejam avaliado o nível de segurança das ferramentas de alcance geral.

Trata-se de uma preocupação ausente na literatura sobre segurança da informação em Javascript. Nela parece existir uma distância entre as inovações introduzidas pelos navegadores e os tópicos sensíveis à comunidade acadêmica. Trazer as duas vertentes em torno de um objeto de pesquisa -- o método investigado por este trabalho -- parece não apenas possível como relevante por sua aplicabilidade imediata, caso se prove suficientemente eficaz.


%!TEX root = Projeto.tex
\section{Método de trabalho}

%COMO ESTE TRABALHO VAI ALCANÇAR O OBJETIVO E CONCRETIZAR AS CONTRIBUIÇÕES?

Para validar a proposta de Encapsulamento por \textit{shadow DOM}, propõe-se uma sequência de tarefas para a exploração do problema e elaboração da solução. O método de trabalho, então, é composto das atividades enumeradas a seguir.


\subsection{Coleta de evidências}
Nesta primeira atividade, os problemas endereçáveis pelo método proposto serão materializados em simulações e casos de teste que evidenciem acessos não autorizados a informações contidas em componentes de páginas da web. Especificamente, as evidências devem provar que é possível efetuar as seguintes ações sem o conhecimento ou consentimento do usuário:

\begin{alineas}
	\item scripts provenientes de domínios diferentes podem observar o conteúdo de páginas da web, incluindo identificações, senhas, códigos de cartão de crédito e números de telefone, desde que essas informações estejam presentes na estrutura da página;
	\item scripts agindo em extensões do navegador podem observar o conteúdo de páginas da web e de seus subcomponentes (\textit{iframes});
	\item scripts de qualquer natureza podem registrar o comportamento do usuário ao interagir com a página, capturando eventos de teclado e de mouse;
	\item scripts de qualquer natureza conseguem interceptar funcionalidades do navegador para extrair informações que transitem pelas interfaces de programação do DOM e de Javascript.
\end{alineas}

Artefatos derivados dos casos de testes, codificados como páginas da web, serão insumo das atividades de avaliação do método proposto. O método será considerado eficaz se os problemas evidenciados forem neutralizados através de sua aplicação.


\subsection{Proposição do método}
O objetivo desta atividade é projetar um método de encapsulamento através do qual desenvolvedores de páginas da web possam definir componentes de HTML imunes ao vazamento de informação por meio de Javascript. O método precisará atender aos seguintes requisitos:

\begin{alineas}
	\item Permitir que qualquer combinação de elementos HTML seja encapsulável em um componente inviolável por scripts externos a si;
	\item Ser compatível com bibliotecas e \textit{frameworks} de desenvolvimento em Javascript, HTML e CSS;
	\item Estabelecer um protocolo de confiança entre si e agentes (scripts) externos;
	\item Sob esse mesmo protocolo, expor uma interface de programação para a leitura e modificação das informações encapsuladas;
	\item Ser compatível com recursos padronizados e não-experimentais de HTML e Javascript.
\end{alineas}

O resultado desta atividade é uma especificação técnica da solução proposta, incluindo pré-requisitos e limitações de uso.


\subsection{Implementação-modelo do método proposto}
Partindo da especificação do método, será construída uma implementação-modelo para fins de validação. Esta atividade tem a finalidade de gerar um artefato de HTML e Javascript compatível com os requisitos estabelecidos pela proposta. Como requisito adicional e específico para esta atividade, a implementação-modelo deve ser transparentemente integrável aos artefatos dos casos de teste, de forma a não produzir nenhuma mudança no comportamento apresentado por eles aos usuários.


\subsection{Avaliação do método}
Nesta tarefa, artefatos que materializam os casos de teste serão modificados para que sejam compatíveis com implementação-modelo do método proposto. O método será assim submetido à avaliação de sua eficácia frente às vulnerabilidades e sua compatibilidade em relação aos pré-requisitos do método. Parte desta tarefa será dedicada à automação dos casos de teste na forma de scripts compatíveis com o \textit{framework} Selenium\footnote{http://www.seleniumhq.org/}. Os scripts de automação serão agregados ao conjunto de artefatos de código derivados deste trabalho.


\subsection{Síntese dos resultados}
A avaliação do método produzirá observações objetivas, como a eficácia contra vulnerabilidades e reflexos no desempenho do navegador, bem como impressões subjetivas como a facilidade de aplicação e a severidade das restrições impostas pelo método. A partir dessas observações será elaborada uma síntese dos resultados alcançados, incluindo uma reflexão sobre a abrangência e utilidade dos artefatos produzidos. As realizações e limitações deste trabalho serão  comparadas com as de trabalhos relacionados, posicionando definitivamente estas contribuições dentro do panorama de segurança da informação.

%!TEX root = Projeto.tex
\section{Organização do trabalho}

%A seção 1, Introdução, fundamenta este trabalho pela caracterização do problema que motivou sua existência, pela definição de um objetivo relevante no domínio do problema, e pelas contribuições que o trabalho se propõe a fazer para o corpo de conhecimento do tema da segurança da informação no navegador com Javascript e HTML.

A seção 2, Estado da Arte, enquadra o tema sob três pontos de vista: (1) das vulnerabilidades derivadas da tecnologia atual, (2) dos recursos implementados pelos navegadores para a detenção de determinados ataques à segurança da informação, e (3) das propostas experimentais para a mitigação de vulnerabilidades. O panorama formado por esses três pontos de vista corresponde ao contexto em que as contribuições deste trabalho estão inseridas.

A seção 3, Proposta, descreve um método para o desenvolvimento de componentes de HTML que mantenham invisíveis, para o restante da página, as informações mantidas ou geradas por esses componentes, ao mesmo tempo em que expõe uma interface de programação baseada em controle do acesso à informação encapsulada. São apresentadas nesta seção a disponibilidade dos recursos necessários para a implementação do método, bem como suas limitações de uso.

Na seção 4, Avaliação, são propostos critérios para a verificação da eficácia do método proposto: disponibilidade nas plataformas de navegação, limites de proteção versus vulnerabilidades mitigadas, e requisitos de funcionamento. A seção se completa com a aplicação desses critérios sobre o método proposto, em comparação com trabalhos embasados pela abordagem de IFC -- o controle de fluxo de informação define, no âmbito do problema, maior granularidade na segurança da informação em Javascript, ao custo da compatibilidade com a base instalada de navegadores.

O conteúdo da seção 5, Conclusões, deriva da reflexão crítica sobre a implementação do método proposto em contraponto aos resultados observados na avaliação qualitativa. Recomendações sobre a aplicação do método, além de oportunidades a serem exploradas por trabalhos futuros, fecham a conclusão dos esforços deste trabalho.

%!TEX root = Projeto.tex
\section{Cronograma}

\begin{figure}[h]
	\centering
	\includegraphics[width=12cm]{cronograma.png}
\end{figure}
%!TEX root = Projeto.tex
\chapter{Fundamentos}

%!TEX root = Projeto.tex
\section{Principais conceitos}
Nesta seção são apresentados os conceitos que embasam este e outros trabalhos relacionados ao tema da segurança da informação em aplicações da web.

\subsection{Segurança da informação}
Segundo \cite{ISO2016}, segurança da informação é um processo com os objetivos de ``preservação da confidencialidade, integridade e disponibilidade da informação''. \cite{Foster1998} elabora esses objetivos, descrevendo a confidencialidade como a condição na qual a informação só pode ser acessada pelos agentes autorizados, integridade como a capacidade de proteger a informação contra modificações não autorizadas, e a disponibilidade como a capacidade de garantir acesso à informação quando necessário; \cite{Foster1998} ainda atribui mais duas características a um sistema de segurança da informação: \textit{accountability} como a possibilidade de se atribuir um agente para cada ação ocorrida dentro do sistema, e \textit{assurance} como o grau de confiabilidade na segurança do sistema em relação aos seus objetivos declarados.

Neste trabalho, qualquer definição de segurança da informação será restrita aos sistemas de informação relacionados com a navegação de usuários através da web: provedores de serviço (\textit{sites}, servidores da web), protocolos de comunicação em rede (HTTP, HTTPS, \textit{web sockets}), navegadores (\textit{browsers}) e os ambientes de execução de Javascript embutidos nos navegadores. Isto delimita a área de conhecimento relevante para este trabalho.

%\subsection{Políticas de controle de acesso}
%Segundo \cite{Goguen1982}, uma política de controle de acesso é necessária para que se estabeleçam quais fluxos de dados serão permitidos em um sistema de informação.

\subsection{Modelos de controle de acesso}
Enquanto a definição dos requisitos de segurança da informação estabelece seus objetivos, os modelos definem os sistemas derivados desses objetivos \cite{Goguen1982}. \cite{Foster1998} menciona diferentes modelos de controle de acesso, categorizados de modo amplo como modelos discricionários (DAC -- \textit{discretionary access control}) e mandatórios (MAC -- \textit{mandatory access control}). Modelos discricionários se baseiam na definição dos relacionamentos de segurança entre agentes e objetos em um sistema, como, por exemplo, a política de que um \script -- a parte \textit{agente} -- não pode iniciar conexões com domínios diferentes do seu próprio -- a parte \textit{objeto}. Modelos discricionários são os mais comumente utilizados para estabelecer mecanismos de segurança nos navegadores. O campo de atuação desses modelos é limitado aos relacionamentos de segurança estabelecidos, e portanto não podem garantir a segurança da informação quando esta ultrapassa o domínio desses relacionamentos. Isto significa, por exemplo, que dados legitimamente obtidos dentro de regras discricionárias pode ser replicado para um contexto não-seguro sem qualquer impedimento derivado do modelo de segurança.

Modelos mandatórios não atribuem explicitamente as regras de controle de acesso aos objetos e agentes de um sistema. Ao invés disso, estabelecem níveis de confidencialidade utilizados para classificar os participantes do sistema de informação, viabilizando o controle dinâmico do trânsito da informação entre os agentes. Num modelo mandatório, o nível de segurança de um dado impede que ele seja obtido ou modificado por agentes com níveis de segurança mais baixos. O controle do fluxo da informação faz dos MACs modelos mais robustos do que os DACs \cite{Foster1998}.

\subsection{Controle do fluxo de informações}
O controle do fluxo de informações (IFC -- \textit{information flow control}) é um mecanismo que atua, em tempo de execução, nos meios de propagação dos valores entre os espaços de armazenamento de um sistema computacional de modo a impedir fluxos não autorizados dos dados \cite{Denning1976}. IFC é um modelo discricionário e baseia-se em \textit{classes de segurança} ``altas'' e ``baixas'', simbolizadas pelas letras \texttt{<h>} e \texttt{<l>}, respectivamente, para indicar graus de confidencialidade das informações e dos seus espaços de armazenamento (\textit{heap}, pilha, redes, dispositivos etc). Operações entre entidades com classes de segurança diferentes, como a cópia do valor de uma variável \texttt{<h>} (confidencial) para a variável \texttt{<l>} (pública), são automaticamente impedidas de prosseguir.

IFC distingue entre fluxos de informação explícitos e implícitos. Um fluxo explícito ocorre quando uma informação classificada como ``alta'' é diretamente copiada para um contexto de classificação ``baixa'', como na listagem de código \ref{Src: jsIFCExplicitFlow}. Em um fluxo implícito, não é a informação em si que transita entre contextos de classificação diferente, mas sim alguma informação derivada dela através da qual seja possível fazer qualquer inferência sobre seu conteúdo. Um exemplo de fluxo implícito encontra-se la listagem \ref{Src: jsIFCImplicitFlow}. Um mecanismo que suporte IFC deve ser capaz de interromper vazamento de informação em ambos os tipos de fluxo.

\lstinputlisting[language=JavaScript,
inputencoding=utf8,
label={Src: jsIFCExplicitFlow},
caption={Vazamento de dados em fluxo explícito de informação}]{codigo/sample02-ifc-implicit.js}

\lstinputlisting[language=JavaScript,
inputencoding=utf8,
label={Src: jsIFCImplicitFlow},
caption={Vazamento de dados em fluxo implícito de informação}]{codigo/sample03-ifc-explicit.js}


\subsection{SOP -- Same Origin Policy}
A política de segurança SOP foi estabelecida para que os navegadores conseguissem dar suporte a páginas com conteúdo proveniente de domínios mistos com um mínimo de segurança contra o vazamento de informação entre esses domínios \cite{Hill2016}. Através desta política, os navegadores podem impedir um conjunto de ataques conhecido como \textit{cross-site resource forgery}, em que um domínio tenta instruir o navegador a fazer requisições para outro domínio em nome do usuário.

O termo \textit{origem} é intercambiável com a expressão \textit{domínio} e ambos representam, para fins desta política, componentes do endereço de URL associado com cada recurso da web -- a saber, o \textit{protocolo}, o \textit{nome do host} e a \textit{porta TCP} de onde o recurso foi transferido \cite{Barth2011}. Os exemplos a seguir representam recursos de mesma origem:

{
	\small \begin{tabular}{|l|c|l|r|}
		\hline 
		Endereço & Protocolo & Nome do \textit{host} & Porta \\ 
		\hline 
		\texttt{http://exemplo.com/} & http & exemplo.com & 80 \\ 
		\hline 
		\texttt{http://exemplo.com:80/} & http & exemplo.com & 80 \\ 
		\hline 
		\texttt{http://exemplo.com/path/file} & http & exemplo.com & 80 \\ 
		\hline 
	\end{tabular}
}


Os endereços a seguir representam recursos de origens diferentes:

{\small
	\begin{tabular}{|l|c|l|r|}
		\hline 
		Endereço & Protocolo & Nome do \textit{host} & Porta \\ 
		\hline 
		\texttt{http://exemplo.com/} & http & exemplo.com & 80 \\ 
		\hline 
		\texttt{http://exemplo.com:8080/} & http & exemplo.com & 8080 \\ 
		\hline 
		\texttt{http://www.exemplo.com/} & http & www.exemplo.com & 80 \\ 
		\hline 
		\texttt{https://exemplo.com:80/} & https & exemplo.com & 80 \\ 
		\hline
		\texttt{https://exemplo.com/} & https & exemplo.com & 443 \\ 
		\hline
		\texttt{http://exemplo.org/} & http & exemplo.org & 80 \\ 
		\hline
	\end{tabular}
}

Segundo a SOP, as atividades derivadas da inclusão de recursos de origens mistas são categorizadas em três ações \cite{Ruderman2017}:

\begin{alineas}
	\item \textbf{Escrita:} atividades deste tipo instruem o navegador para que ocorra alguma forma de navegação entre páginas, o que inclui a interação com \textit{links}, redirecionamento e submissão de formulários. Em geral SOP não restringe este tipo de ação;
	\item \textbf{Incorporação:} SOP permite que recursos incorporados à página tenham origens mistas. Isto significa que é possível a inclusão de imagens, vídeos, \scripts e do elemento \texttt{<iframe>}, entre outros, provenientes de origens mistas e dentro de uma mesma página.
	\item \textbf{Leitura:} atividades de leitura permitiriam que o conteúdo dos recursos carregados pudesse ser consultado entre origens. SOP permite que um subconjunto de funcionalidades de leitura possam ocorrer entre domínios diferentes.
\end{alineas}

Um aspecto importante da SOP é o tratamento dado a \scripts incorporados. Quando uma página inclui um \script proveniente de outras origens, por exemplo pelo uso de uma CDNs (\textit{content distribution networks}), esses \scripts são executados em contexto da origem do documento em que eles foram incorporados. Isto permite, por exemplo, que \textit{frameworks} populares como jQuery e Angular.js possam ser disponibilizados em CDNs sem perder funcionalidades importantes, como a capacidade de iniciar chamadas assíncronas pela técnica AJAX. Esta concessão da SOP, porém, abre a possibilidade de que esses scripts, se adulterados, executem atividades maliciosas sem impedimentos.

\subsection{CSP -- Content Security Policy}
CSP foi criada como um complemento à SOP, elevando a capacidade do navegador de servir como plataforma razoavelmente segura para composição de aplicações \textit{mashup} ao estabelecer um protocolo para o compartilhamento de dados entre os componentes da página que residam em domínios diferentes. CSP define um conjunto de diretivas (codificadas como cabeçalhos HTTP) para a definição de \textit{whitelists} -- o conjunto de origens confiáveis em um dado momento -- pelas quais navegador e provedores de conteúdo estabelecem o controle de acesso e o uso permitido de recursos embutidos como \scripts, folhas de estilos, imagens e vídeos, entre outros. Através desse protocolo, ataques de XSS que podem ser neutralizados desde que todos os componentes na página sejam aderentes à mesma política de CSP.

\subsection{CORS -- Cross-Origin Resource Sharing}
Assim como a CSP, o mecanismo CORS \cite{W3C:CORS} complementa a SOP estabelecendo um conjunto de diretivas (cabeçalhos HTTP) para a negociação de acesso via Ajax/XHR a recursos hospedados em domínios diferentes. CORS determina que exista um vínculo de confiança entre navegadores e provedores de conteúdo, dificultando vazamento de informação ao mesmo tempo em que flexibiliza as funcionalidades das APIs. O uso de CORS permite que os autores de componentes e desenvolvedores de aplicações \textit{mashup} determinem o grau de exposição que cada conteúdo pode ter em relação aos outros conteúdos incorporados.

CSP e CORS são recomendações do comitê W3C \cite{W3C:CSP} \cite{W3C:CORS}, sendo incorporados por todos os navegadores relevantes desde 2016 \cite{CanIUse:CSP} \cite{CanIUse:CORS}.

\subsection{Vulnerabilidades}
Violações de privacidade são possíveis nos navegadores por causa da natureza dinâmica da linguagem Javascript e de sua ausência de restrições de segurança em tempo de execução \cite{Jang2010}. Seus usuários estão expostos a ataques sutis com objetivos diversos como roubar \textit{cookies} e \textit{tokens} de autorização, redirecionar o navegador para sites falsos (\textit{phishing}), observar o histórico de navegação e rastrear o comportamento do usuário através dos movimentos do ponteiro do mouse e eventos de teclado. Para que \scripts mal-intencionados sejam incorporados a páginas benignas, \textit{hackers} fazem uso de vulnerabilidades como \textit{cross-site scripting (XSS)} \cite{OWASP:XSS} e comprometimento de extensões \cite{Heule2015_Most_Dangerous_Code} do navegador.


\subsubsection{Compartilhamento do ambiente de execução}
Código \textit{inline} ou \scripts baixados pelas páginas da web são executados com os mesmos privilégios e mesmo nível de acesso à estrutura de documento do navegador, o chamado DOM (\textit{document object model}) \cite[p. 2-3]{DeRyck2012}, não importando o domínio de origem dos \scripts. Uma demonstração do problema pode ser exemplificada na figura \ref{Fig: diagrama01} e listagem de código \ref{Src: webPageMultiOrigin}. Nesse exemplo, um \script tido como benigno é incorporado a uma página web a partir de um domínio de CDN (\textit{content delivery network}), diferente daquele da aplicação que efetivamente publica a página. O servidor da página, pelo protocolo CORS, sinaliza ao navegador que o domínio da CDN é confiável. O \script externo pode, então, iniciar requisições ao seu domínio de origem -- uma consequência desejada pelos autores da página, pois o \script depende desse acesso para efetuar suas funções.

\begin{figure}
	\centering
	\includegraphics[width=10cm]{diagramas/diagrama01.pdf}
	\caption{Aplicação web composta por conteúdo proveniente de duas origens.}
	\label{Fig: diagrama01}
	
	\lstinputlisting[language=html,
	inputencoding=utf8,
	label={Src: webPageMultiOrigin},
	caption={[Página HTML incorporando \script de outra origem]Incorporação de \script de outra origem (linha \ref{lstCdnScript})}]{codigo/sample01-leaking-script.html}
\end{figure}

Em momento posterior, o \script servido pelos servidores da CDN é substituído por código malicioso que, além de efetuar as funções do \script benigno, captura o conteúdo da página armazenado no DOM \ref{Fig: diagrama02}. O \script pode buscar informações específicas e potencialmente sensíveis como identificação do usuário, senhas e endereços. Por causa da autorização concedida pelo protocolo CORS, o código mal intencionado tem a chance de transmitir o conteúdo capturado para um serviço anômalo.

\begin{figure}
	\centering
	\includegraphics[width=10cm]{diagramas/diagrama02.pdf}
	\caption{Domínio de CDN comprometido, capturando informações do usuário.}
	\label{Fig: diagrama02}
\end{figure}

Acessar, capturar e modificar informações contidas no DOM também são efeitos de extensões do navegador. Mas, diferentemente dos \scripts incorporados em páginas, extensões são executadas em modo privilegiado e podem afetar todas as páginas carregadas pelo navegador, não sendo confinadas a domínios específicos. Extensões como as do Google Chrome são publicadas exclusivamente em site específico e protegido, mas não é impossível que o código fonte de extensões seja descaracterizado e publicado pela ação de \textit{hackers} \cite{Spring2017}, afetando a todos os usuários que atualizarem a extensão -- um processo automático por padrão \cite{Google2017}.


\subsubsection{Cross-Site Scripting (XSS)}
Em Javascript, todos os recursos de código carregados dentro de uma mesma página possuem os mesmos privilégios de execução. Ataques do tipo \textit{cross-site scripting} tiram proveito dessa característica para injetar código malicioso em contextos onde seja possível observar e retransmitir informação sigilosa como \textit{cookies} do usuário, endereço do navegador, conteúdo de formulários, ou qualquer outra informação mantida pelo DOM.

O emprego de medidas para prevenção de ataques XSS \cite{OWASP:XSS-CheatSheet} não elimina riscos inerentes à tecnologia do navegador. Uma vez que componentes incorporados, como anúncios e \textit{players} de mídia, conseguem carregar \scripts tidos como confiáveis dinamicamente, um único trecho de código comprometido pode colocar informações em risco sem qualquer interferência dos dispositivos de segurança.

%\subsubsection{Sobrescrita do DOM} DOM CLOBBERING

\subsubsection{Comprometimento de extensões}
Os mecanismos de extensibilidade oferecidos pelos navegadores melhoram a funcionalidade da web para os usuários, e o código de que são feitos é executado com privilégios mais elevados do que o dos \scripts incorporados pelos \textit{sites}. Por isso, os usuários precisam confirmar ao navegador que aceitam que uma extensão seja instalada, sendo informados a respeito dos privilégios que a extensão pretende utilizar. O fato de que esse processo precisa se repetir a cada vez que uma extensão necessita de um conjunto de privilégios diferente faz com que os desenvolvedores optem por solicitar, de antemão, uma gama de privilégios maior que a estritamente necessária \cite{Heule2015_Most_Dangerous_Code}.

Uma extensão que tiver sido comprometida (por exemplo, ao usar \scripts de terceiros que, por sua vez, tenham sido redirecionados ou adulterados) terá assim poder para ler e transmitir todo o conteúdo carregado e exibido pelo navegador, com o potencial de causar os mesmos efeitos observados em um ataque XSS, mas em escopo e poder aumentados, já que poderiam afetar todas as páginas abertas e todas as APIs publicadas pelo navegador.


%!TEX root = Projeto.tex

%VALE A PENA OLHAR O QUE PODE SER APROVEITADO DOS TRABALHOS ENUMERADOS COMO "SANDBOXING" NO ARTIGO SOBRE COWL.

\section{Estado da arte}

Esta seção destaca os trabalhos que exerceram maior influência na concepção deste trabalho e na delimitação da proposta.

\subsubsection{Security of web mashups: A survey \cite{DeRyck2012}}
O artigo é motivado pelos requisitos de segurança de aplicações web que agregam conteúdo ativo de origens distintas (\textit{web mashups}). Os autores definem um conjunto de categorias de requisitos não-funcionais de segurança e avaliam a conformidade desses requisitos versus funcionalidades do navegador. O critério de classificação estabelecido posiciona as diversas abordagens em quatro graduações que vão desde a separação total de componentes até sua integração completa.

\textbf{Contribuição.} O artigo contribui com a enumeração de requisitos que uma solução voltada à segurança da informação deve atender. Algumas das tecnologias mencionadas podem ter se tornado obsoletas ou de alcance limitado desde que o artigo foi escrito, o que não invalida o resultado pretendido pelos autores, que é considerado ``estado da arte'' \cite{Hedin2014} em pesquisa sobre segurança de aplicações de composição baseadas em Javascript.


\subsubsection{Toward Principled Browser Security \cite{Yang2013}}
Os autores analisam os mecanismos tradicionais SOP, CORS e CSP para avaliar suas heurísticas e políticas de segurança que, em troca de flexibilidade para o desenvolvedor de aplicações web, abrem diversas janelas para o vazamento de dados. Partindo dessa condição, os autores propõem um modelo baseado em controle do fluxo da informação capaz de suportar todas as políticas associadas a esses mecanismos, sem apresentar as mesmas vulnerabilidades.

\textbf{Contribuição.} O artigo contribui pela reinterpretação dos mecanismos de segurança tradicionais à luz do IFC, revelando algumas das suas inconsistências e concessões em prol de funcionalidades que podem ser exploradas em ataques. Esse conhecimento fornece insumos para a definição de cenários de testes a serem desenvolvidos neste trabalho.


\subsubsection{JSFlow: Tracking information flow in JavaScript and its APIs \cite{Hedin2014}}
O trabalho, uma continuação de outro de mesma autoria \cite{Hedin2012}, é composto de duas partes: primeiro, os autores descrevem o panorama geral do corpo de conhecimento em segurança da informação no software navegador, detalhando as vulnerabilidades mais comuns; e em segundo, apresentam o projeto JSFlow, que adiciona a capacidade de IFC ao navegador. O trabalho é concluído com um teste da eficácia do projeto.

\textbf{Contribuição.} O projeto JSFlow demonstra alguns dos desafios de uma implementação de IFC no navegador, especialmente a possibilidade de ocorrência dos ``falsos positivos'', ocasiões em que acessos legítimos são impedidos pelo sistema. Em tal situação, o desenvolvedor estaria diante de um impasse e talvez preferisse adotar uma abordagem discricionária, como é a proposta deste trabalho.


\subsubsection{Protecting Users by Confining JavaScript with COWL \cite{Stefan2014}}
Em concordância com o artigo associado \cite{Yang2013}, os autores argumentam que, face às dificuldades que os desenvolvedores encontram para aderir aos mecanismos tradicionais SOP, CSP e CORS, acaba-se optando pela funcionalidade em detrimento da segurança. Isto se manifesta em extensões de navegador solicitando mais permissões do que o necessário, em \textit{web mashups} que requerem autorizações desnecessárias para o usuário, e em notificações de segurança tão constantes que se tornam efetivamente invisíveis. Entendendo que o estado-da-arte da análise do fluxo de informações em navegador é deficiente -- seja porque as ferramentas são incompletas ou porque degradam desempenho --, os autores apresentam o projeto COWL, um navegador construído sobre o software Firefox que implementa, experimentalmente, o controle do fluxo da informação.

\textbf{Contribuição.} O navegador COWL será utilizado como participante de cenários de teste a serem explorados neste trabalho, para que seja avaliada a efetividade de uma implementação de IFC como meio de mitigar as vulnerabilidades identificadas.



\subsubsection{Information Flow Control for Event Handling and the DOM in Web Browsers \cite{Rajani2015}}
O artigo explora vazamento de informação em \scripts acionados por eventos do DOM, demonstrando que esse é um efeito de como os navegadores disparam e propagam eventos. Os autores apontam as particularidades da propagação de eventos e suas implicações para o controle do fluxo da informação, implementando, no navegador WebKit, um mecanismo de IFC imune a vazamento de informação derivado do disparo de eventos.

\textbf{Contribuição.} Este artigo coloca em pauta a segurança da informação no âmbito dos eventos do DOM, uma forma sutil de materializar o problema central deste trabalho. Também contribui com a definição de uma máquina de estados derivada do comportamento do navegador na ocorrência de um evento, útil para a concepção de casos de testes.


\subsubsection{The Most Dangerous Code in the Browser \cite{Heule2015_Most_Dangerous_Code}}
O artigo apresenta o desafio, ainda não superado, das vulnerabilidades derivadas do modelo de confiança que os navegadores utilizam para instalar, acionar e atualizar software de extensão. Os autores demonstram como uma extensão pode comprometer a segurança da informação, propondo um modelo de controle mandatório de acesso para mitigar oportunidades de vazamento de dados.

\textbf{Contribuição.} Este artigo mapeia as vulnerabilidades associadas às interações entre o DOM e as extensões do navegador. Esse conhecimento é fundamental para que seja possível definir requisitos e cenários de testes que envolvam extensões.


% ----------------------------------------------------------
% ELEMENTOS PÓS-TEXTUAIS
% ----------------------------------------------------------
\postextual
%!TEX root = Projeto.tex
% ----------------------------------------------------------
% Referências bibliográficas
% ----------------------------------------------------------
\setboolean{isBib}{true} % Obriga o titulo das referências à esquerda

\setlength\bibitemsep{12pt} % Espaço de uma linha entre as referencias

\bibliography{Projeto}

\setboolean{isBib}{false} % Desobriga o titulo das referências à esquerda

% ----------------------------------------------------------
% Apêndices
% ----------------------------------------------------------

%\apendices
%\partapendices
%\chapter{Lorem ipsum dolor sit amet}

%\lipsum[21-26]

% ----------------------------------------------------------
% Anexos
% ----------------------------------------------------------

% \anexos
% \partanexos
% \chapter{Padrões de projeto para mitigação de riscos}

\section{}
%\lipsum[26-32]




\end{document}
