%!TeX program = xelatex
%
% O MODELO DESTE DOCUMENTO É abntex2-modelo-trabalho-academico.tex
% HÁ MAIS SEÇÕES NO MODELO DO QUE NESTE ARQUIVO AQUI.
% É PARA CONSTRUIR AOS POUCOS.
%
%
\documentclass[
	% -- opções da classe memoir --
	12pt,				% tamanho da fonte
	openright,			% capítulos começam em pág ímpar (insere página vazia caso preciso)
	twoside,			% para impressão em recto e verso. Oposto a oneside
	a4paper,			% tamanho do papel. 
	% -- opções da classe abntex2 --
	%chapter=TITLE,		% títulos de capítulos convertidos em letras maiúsculas
	%section=TITLE,		% títulos de seções convertidos em letras maiúsculas
	%subsection=TITLE,	% títulos de subseções convertidos em letras maiúsculas
	%subsubsection=TITLE,% títulos de subsubseções convertidos em letras maiúsculas
	% -- opções do pacote babel --
	english,			% idioma adicional para hifenização
	brazil				% o último idioma é o principal do documento
	]{abntex2}

\usepackage{_estilos}

% ====================================================================================================================================
% ---
% Informações de dados para CAPA e FOLHA DE ROSTO
% ---
\titulo{Método para encapsulamento da informação em aplicações de HTML e Javascript}
\autor{Humberto Borgas Bulhões}
\local{São Paulo}
\data{2017}
\mes{Fevereiro}
\orientador{Marcelo Novaes de Rezende}
\instituicao{\mbox{Instituto} de Pesquisas Tecnológicas do \mbox{Estado} de São Paulo}
\tipotrabalho{Dissertação de Mestrado}
\curso{Engenharia da Computação: Engenharia de Software}
\concentracao{Área de Concentração: Engenharia de \mbox{Software}}
\tipotrabalho{Dissertação de Mestrado}
% O preambulo deve conter o tipo do trabalho, o objetivo, 
% o nome da instituição e a área de concentração 
\preambulo{Exame de Qualificação apresentado ao
	Instituto de Pesquisas Tecnológicas do
	Estado de São Paulo – IPT, como parte
	dos requisitos para a obtenção do título
	de mestre em Engenharia de
	Computação.}
% ---


% ====================================================================================================================================
% ---
% compila o indice
% ---
%\makeindex
% ---

% ====================================================================================================================================
% ----
% Início do documento
% ----
\begin{document}

\chapterstyle{iptabntex2}			% Estilo de capitulo adotado pelo IPT

% Retira espaço extra obsoleto entre as frases.
\frenchspacing

% ----------------------------------------------------------
% ELEMENTOS PRÉ-TEXTUAIS
% ----------------------------------------------------------
\pretextual
%!TEX root = Projeto.tex
\imprimircapa
\imprimirfolhadeaprovacao
\imprimirfolhaderosto

% ---
% Inserir a ficha bibliografica
% ---

% Isto é um exemplo de Ficha Catalográfica, ou ``Dados internacionais de
% catalogação-na-publicação''. Você pode utilizar este modelo como referência.
% Porém, provavelmente a biblioteca da sua universidade lhe fornecerá um PDF
% com a ficha catalográfica definitiva após a defesa do trabalho. Quando estiver
% com o documento, salve-o como PDF no diretório do seu projeto e substitua todo
% o conteúdo de implementação deste arquivo pelo comando abaixo:
%
% \begin{fichacatalografica}
%     \includepdf{fig_ficha_catalografica.pdf}
% \end{fichacatalografica}
\begin{fichacatalografica}
	\vspace*{\fill}					% Posição vertical
	\hrule							% Linha horizontal
	\begin{center}					% Minipage Centralizado
	\begin{minipage}[c]{12.5cm}		% Largura

	\imprimirautor

	\hspace{0.5cm} \imprimirtitulo  / \imprimirautor. --
	\imprimirlocal, \imprimirdata-

	\hspace{0.5cm} \pageref{LastPage} p. : il. (algumas color.) ; 30 cm.\\

	\hspace{0.5cm} \imprimirorientadorRotulo~\imprimirorientador\\

	\hspace{0.5cm}
	\parbox[t]{\textwidth}{\imprimirtipotrabalho~--~\imprimirinstituicao,
	\imprimirdata.}\\

	\hspace{0.5cm}
%		1. Palavra-chave1.
%		2. Palavra-chave2.
%		I. Orientador.
%		II. Universidade xxx.
%		III. Faculdade de xxx.
%		IV. Título\\

	\hspace{8.75cm} CDU 02:141:005.7\\

	\end{minipage}
	\end{center}
	\hrule
\end{fichacatalografica}
% ---



% ---
% Dedicatória
% ---
%\begin{dedicatoria}
%\vspace*{\fill}
%\OnehalfSpacing
%%\centering
%%\noindent
%Dedico este trabalho...
%
%
%\vspace*{\fill}
%\end{dedicatoria}
% ---

% ---
% Agradecimentos
% ---
%\begin{agradecimentos}
%\vspace*{\fill}
%\OnehalfSpacing
%Gostaria de agradecer...
%
%
%\vspace*{\fill}
%\end{agradecimentos}
% ---

% ---
% Epígrafe
% ---
%\begin{epigrafe}
%    \vspace*{\fill}
%	\begin{flushright}
%		\textit{``Não vos amoldeis às estruturas deste mundo, \\
%		mas transformai-vos pela renovação da mente, \\
%		a fim de distinguir qual é a vontade de Deus: \\
%		o que é bom, o que Lhe é agradável, o que é perfeito.\\
%		(Bíblia Sagrada, Romanos 12, 2)}
%	\end{flushright}
%\end{epigrafe}
% ---


%!TEX root = Projeto.tex
\newpage
\begin{resumo}
\normalsize

Impulsionados por recursos avançados de HTML e Javascript, os programas navegadores da web tornaram-se plataformas de desenvolvimento e publicação de aplicativos. O aumento em escopo de funcionalidade dos navegadores, porém, vem acompanhado de vulnerabilidades de segurança da informação que podem expor dados do usuário. Muitas dessas vulnerabilidades se manifestam de formas sutis, mantendo-se fora do alcance das políticas de segurança padronizadas e adotadas pela indústria, e afetando a confidencialidade da informação que trafega pelo navegador. Para neutralizar esse problema diversas iniciativas experimentais têm sido propostas, sem que tenham encontrado, até o momento, adoção em massa. Ao mesmo tempo, novos recursos de programação têm sido incorporados aos navegadores, e alguns deles, quando utilizados em conjunto, levantam a possibilidade de implementação do encapsulamento da informação, tornando-a invisível a agentes não autorizados. O objetivo deste trabalho é propor um método que, fundamentado em capacidades de programação padronizadas e difundidas, implemente o encapsulamento da informação em HTML e Javascript, o que poderia tornar realidade alguns dos requisitos de segurança que apenas abordagens experimentais têm conseguido assegurar. Para tanto, este trabalho apresentará uma especificação desse método, uma implementação modelo e uma avaliação da eficácia do método em relação a determinados casos de teste que evidenciam problemas de confidencialidade da informação.

\vspace{\onelineskip}

\noindent
\textbf{Palavras-chave:} \imprimirpalavraschave
\end{resumo}

% resumo em inglês
%\begin{resumo}[Abstract]
%%\begin{otherlanguage*}{english}
%Resumo da dissertação em inglês.
%
%
%\vspace{\onelineskip}
%
%\noindent
%\textbf{Keywords:} \imprimirkeyword   
%%\end{otherlanguage*}
%\end{resumo}

% ---
% inserir lista de ilustrações
% ---
{\SingleSpacing
\pdfbookmark[0]{\listfigurename}{lof}
\listoffigures*
\cleardoublepage
}
% ---

% ---
% inserir lista de trechos de código fonte
% ---
{\SingleSpacing
	\pdfbookmark[0]{\lstlistlistingname}{lol}
	\lstlistoflistings
	\cleardoublepage
}
% ---

% ---
% inserir lista de quadros
% ---
%\pdfbookmark[0]{\listofquadrosname}{loq}
%\listofquadros*
%\cleardoublepage
% ---

% ---
% inserir lista de tabelas
% ---
%{\SingleSpacing
%\pdfbookmark[0]{\listtablename}{lot}
%\listoftables*
%\cleardoublepage
%}
% ---

% ---
% inserir lista de abreviaturas e siglas
% ---
%
\begin{siglas}
  \item[CORS] Cross-Origin Resource Sharing
  \item[CSRF] Cross-Site Request Forgery
  \item[DAC] Discretionary Access Control
  \item[DOM] Document Object Model
  \item[IFC] Information Flow Control
  \item[JSON] JavaScript Object Notation
  \item[MAC] Mandatory Access Control
  \item[SOP] Same Origin Policy
  \item[XSS] Cross-Site Scripting
\end{siglas}

% ---

% ---
% inserir lista de símbolos
% ---
%\begin{simbolos}
%  \item[$ \Gamma $] Letra grega Gama
%  \item[$ \Lambda $] Lambda
%  \item[$ \zeta $] Letra grega minúscula zeta
%  \item[$ \in $] Pertence
%\end{simbolos}
% ---

% ---
% inserir o sumario
% ---
{\SingleSpacing
	\cleardoublepage
	\pdfbookmark[0]{\contentsname}{toc}
	\tableofcontents*
	\cleardoublepage
}
% ---


% ----------------------------------------------------------
% ELEMENTOS TEXTUAIS
% ----------------------------------------------------------
\textual
\OnehalfSpacing

%!TEX root = Projeto.tex
\chapter{Introdução}

 %!TEX root = Projeto.tex
\section{Motivação}

%ESTÁ MUITO CONFUSO. MELHOR COMEÇAR DIRETO AO PONTO COMO O DEIAN. DEIXAR EVIDENTE A DISTÂNCIA ENTRE O ESTADO DA ARTE E O ESTADO DAS FERRAMENTAS, E POR QUE ISSO COMPLICA A VIDA DE DESENVOLVEDORES E USUÁRIOS. ESCREVER PENSANDO QUE O OBJETIVO DO TRABALHO É ""AVALIAR"" UMA ALTERNATIVA AO ESTADO DA ARTE, SEM EFETIVAMENTE DIZÊ-LO AQUI.

É difícil desenvolver aplicações seguras para a web. Mesmo sob a defesa de \textit{firewalls}, protocolos de segurança, software antivírus e atualizações periódicas dos navegadores, as informações dos usuários permanecem fundamentalmente expostas ao acesso indevido por \textit{scripts} mal-intencionados ou mal-escritos. Essa condição se expressa de forma crítica por meio do chamado conteúdo ativo, recurso do navegador que viabiliza a existência de aplicações interativas nas linguagens HTML e Javascript. Criar um \textit{script} com a intenção de ler o conteúdo potencialmente sigiloso de uma página da web e revelá-lo a terceiros não-autorizados é uma tarefa que exige pouca habilidade e que pode passar despercebida pelo aparato de segurança disponível.

Exemplos de manipulação maliciosa de scripts incluem sequestro de redes de distribuição de conteúdo (CDN, \textit{content distribution network}) \cite{Dorfman2013}, injeção de scripts por extensões, por vírus ou por redirecionamento \cite{Kinlan2015}, e ainda a inclusão de código não verificado pelo publicador \cite{Vanunu2016}. Tais tipos de ataques não tiram proveito de defeitos dos navegadores; ao invés disso, funcionam de acordo com os mecanismos de segurança da informação padronizados pela indústria. Isto ocorre porque parte da tecnologia que dá suporte ao conteúdo ativo é fundamentalmente limitada no nível de segurança da informação que pode oferecer, especialmente quando é considerado o rico ambiente de execução proporcionado pelo software navegador. Nele, a confidencialidade da informação é vulnerável a agentes maliciosos que operam em dois âmbitos:

\begin{alineas}
	\item Componentes de \textit{script} incorporados às páginas da web são executados com os mesmos privilégios e mesmo nível de acesso à página como um todo \cite[p. 2-3]{DeRyck2012}, sejam eles benignos ou não. O navegador oferece diretivas de segurança limitadas para conter o vazamento de informação entre componentes, uma vez que não há um meio de especificar relações de confiança entre eles \cite{Jang2010}.
	\item \textit{Plugins} e extensões do navegador têm nível de acesso mais elevado aos recursos de sistema do que aquele definido para componentes de páginas da web, permitindo que \textit{plugins} tomem controle de funcionalidades como gerenciamento de conexões de rede, eventos de navegação e o DOM. Ainda que dependam da permissão explícita do usuário no momento de sua instalação, os \textit{plugins} e extensões podem se comportar como agentes de vazamento de informação de forma sutil e não necessariamente proposital \cite{Heule2015_Most_Dangerous_Code}.
\end{alineas}

Na raiz das vulnerabilidades está a forma inconsistente e parcial com que a linguagem Javascript e as APIs do navegador tratam o isolamento entre \textit{scripts} e dados. A caracterização e a solução desse problema fazem parte de um campo de pesquisas ativo \cite{Stefan2014}, \cite{Hedin2014}, \cite{Bichhawat2014}, \cite{Magazinius2014} e em busca por padronização \cite{W3C:WebAppSec}, mas que ainda não resultou em práticas, ferramentas e protocolos de amplo alcance pois dependem da adoção de políticas de segurança experimentais \cite{Hedin2014}, \cite{Bichhawat2014} e potencialmente degradantes de desempenho \cite[p. 14]{Stefan2014} por parte dos desenvolvedores de navegadores.

No campo das tecnologias experimentais estão as estratégias baseadas no controle do fluxo da informação (IFC, \textit{information flow control}). IFC, inicialmente descrito por \cite{Denning1976}, estabelece que cada espaço de armazenamento em um programa -- arquivos, segmentos de memória, conexões de rede ou variáveis, por exemplo -- seja rotulado por uma classe de segurança. Em função disso, o trânsito de informação entre espaços de armazenamento deve ser monitorado e, eventualmente, interrompido quando os rótulos da origem e do destino da informação não forem compatíveis. IFC é um mecanismo inexistente nas implementações da linguagem Javascript dos navegadores da web; implementá-lo em uma linguagem dinâmica como Javascript significa introduzir uma checagem de rótulos a cada leitura e escrita em objetos mutáveis \cite[p.3]{Heule2015_IFC_Inside}, acarretando perdas significativas na velocidade da execução de \textit{scripts}.

Os desenvolvedores de aplicações em conteúdo ativo, por sua vez, têm limitadas opções para a publicação de \textit{sites} e páginas web seguras. Ainda que sigam as práticas recomendadas para a neutralização dos ataques mais comuns à segurança da informação \cite{W3C:CORS}, \cite{W3C:SOP}, \cite{W3C:CSP}, o conteúdo das páginas da web permanece acessível aos \textit{scripts} incorporados e às extensões do navegador.





%!TEX root = Projeto.tex
\section{Objetivo}

%A MOTIVAÇÃO DEVE TER ESCLARECIDO QUE EXISTEM PROBLEMAS DE INFOSEC DIFÍCEIS DE SE CONTORNAR COM AS DIRETIVAS TRADICIONAIS. NESTA SEÇÃO DEVERÁ FICAR CLARO QUE UMA ABORDAGEM DIFERENTE ""PODE"" SER AVALIADA, E QUE É OBJETIVO DESTE TRABALHO FAZER ESSA AVALIAÇÃO.

%Para formular um Objetivo realista, é preciso ter uma hipótese (hipo-tese).
%Hipótese: uma afirmação que deverá ser testada (parece razoável, baseando-me no que li, ou no que observei, que é possível fazer X...). É preciso justificar a hipótese.

O objetivo deste trabalho é propor uma abordagem para o confinamento de informação usando recursos padronizados de HTML e Javascript. A proposta deverá permitir ao desenvolvedor a delimitação de regiões do DOM cujo conteúdo seja opaco para scripts incorporados e extensões do navegador.

A efetividade da proposta deve ser avaliada segundo requisitos não funcionais enumerados por \cite{DeRyck2012}:

\begin{alineas}
	\item Efetividade da separação do DOM: a estrutura de documento mantida em regiões ocultas pela abordagem proposta é separada do restante da página;
	\item Efetividade do isolamento de scripts: scripts carregados em regiões ocultas não poderão sofrer influência de scripts externos a essas regiões;
	\item Confidencialidade: a informação mantida em regiões ocultas só poderá ser lida por scripts especialmente criados para esse fim pelo desenvolvedor da aplicação;
	\item Integridade: a informação mantida em regiões ocultas só poderá ser modificada por scripts especialmente criados para esse fim pelo desenvolvedor da aplicação;
	\item Autenticidade: o protocolo de comunicação com regiões ocultas deve suportar apenas participantes que confiem uns aos outros.
\end{alineas}

Requisitos funcionais da proposta devem satisfazer sua compatibilidade com os navegadores modernos, não-experimentais:

\begin{alineas}
	\item Permitir que qualquer combinação de elementos HTML seja encapsulada em regiões ocultas;
	\item Ser compatível com bibliotecas e \textit{frameworks} de desenvolvimento em Javascript, HTML e CSS;
	\item Expor uma interface de programação para a leitura e modificação das informações contidas em regiões ocultas.
\end{alineas}

Como objetivo secundário, será implementada uma prova de conceito que valide os requisitos estabelecidos.

%!TEX root = Projeto.tex
\section{Resultados esperados e contribuições}

%A SEÇÃO "OBJETIVO" ESTABELECEU QUE O TRABALHO VAI TRATAR DA PROTEÇÃO (OU NÃO) QUE UMA TECNOLOGIA COMO "SHADOW DOM" OFERECE AOS USUÁRIOS DE PÁGINAS WEB. NESTA SEÇÃO: QUE RESULTADOS SÃO ESPERADOS DESSE ESTUDO [INCLUIR: REVISÃO/LANDSCAPE (SIM) DO ESTADO DA ARTE E DAS VULNERABILIDADES DOS NAVEGADORES] [INCLUIR: MAPEAMENTO DOS ASPECTOS QUALITATIVOS DE UMA SOLUÇÃO DE INFOSEC PARA JAVASCRIPT EM NAVEGADOR]? COMO ESSES RESULTADOS SÃO IMPORTANTES PARA O AVANÇO NO ESTADO DA ARTE? COMO OS ARTEFATOS DERIVADOS DO TRABALHO PODEM CONTRIBUIR PARA A SEGURANÇA DA INFORMAÇÃO EM PÁGINAS DA WEB?

Este trabalho contribui com a investigação do potencial de inviolabilidade da informação oferecido por tecnologias de ampla disponibilidade, como \textit{shadow DOM} \cite{W3C:ShadowDOM} e \textit{iframes}, em relação a uma abordagem de referência baseada em IFC (IFC -- \textit{information flow control}). O referencial é relevante pois IFC, que fundamentalmente redefine o fluxo de informação em Javascript, ofereceria o nível mais alto de segurança se fosse integrada por padrão às APIs dos navegadores. Aos desenvolvedores e usuários é oportuno, portanto, que sejam avaliado o nível de segurança das ferramentas de alcance geral.

Trata-se de uma preocupação ausente na literatura sobre segurança da informação em Javascript. Nela parece existir uma distância entre as inovações introduzidas pelos navegadores e os tópicos sensíveis à comunidade acadêmica. Trazer as duas vertentes em torno de um objeto de pesquisa -- o método investigado por este trabalho -- parece não apenas possível como relevante por sua aplicabilidade imediata, caso se prove suficientemente eficaz.


%!TEX root = Projeto.tex
\section{Método de trabalho}

%COMO ESTE TRABALHO VAI ALCANÇAR O OBJETIVO E CONCRETIZAR AS CONTRIBUIÇÕES?

Os objetivos deste trabalho serão o produto de uma sequência de atividades dedicadas à exploração do problema e elaboração da solução. O método de trabalho, então, é composto das atividades enumeradas a seguir.


\begin{alineas}
	\item \textbf{Levantamento bibliográfico}
	Esta atividade realiza-se pela pesquisa de trabalhos relacionados à segurança da informação no software navegador, incluindo contribuições relevantes que deram embasamento a esses trabalhos.
	
	\item \textbf{Coleta de evidências}
	Nesta atividade, os problemas-alvo motivadores deste trabalho serão materializados em simulações e casos de teste. As evidências deverão provar que é possível efetuar as seguintes ações sem o conhecimento ou consentimento do usuário:
	
	\begin{alineas}
		\item scripts provenientes de domínios diferentes podem observar o conteúdo de páginas da web, incluindo identificações, senhas, códigos de cartão de crédito e números de telefone, desde que essas informações estejam presentes no DOM;
		\item scripts agindo em extensões do navegador podem observar o conteúdo de páginas da web e de seus \textit{iframes};
		\item scripts de qualquer natureza podem registrar o comportamento do usuário ao interagir com a página, capturando eventos de teclado e de mouse;
		\item scripts de qualquer natureza conseguem interceptar APIs e com isso extrair informações que transitem pelas interfaces de programação do DOM e da linguagem Javascript.
	\end{alineas}
	
	\item \textbf{Proposição}
	O objetivo desta atividade é elaborar um método para que os objetivos do trabalho sejam alcançados, em aderência aos requisitos funcionais e não-funcionais estabelecidos.

	\item \textbf{Implementação}
	Esta atividade tem a finalidade de produzir um componente de HTML e Javascript compatível com os requisitos estabelecidos pela proposta.
		
	\item\textbf{Avaliação do método}
	Nesta tarefa, o componente implementado será submetido à avaliação de sua eficácia frente às vulnerabilidades, e de sua compatibilidade em relação aos requisitos do método.
	
	\item \textbf{Síntese dos resultados}
	A partir das observações produzidas na atividade de avaliação, será elaborada uma síntese dos resultados alcançados em contraste com os objetivos desta proposta.
\end{alineas}


%!TEX root = Projeto.tex
\section{Organização do trabalho}

%A seção 1, Introdução, fundamenta este trabalho pela caracterização do problema que motivou sua existência, pela definição de um objetivo relevante no domínio do problema, e pelas contribuições que o trabalho se propõe a fazer para o corpo de conhecimento do tema da segurança da informação no navegador com Javascript e HTML.

A seção 2, Estado da Arte, enquadra o tema sob três pontos de vista: (1) das vulnerabilidades derivadas da tecnologia atual, (2) dos recursos implementados pelos navegadores para a detenção de determinados ataques à segurança da informação, e (3) das propostas experimentais para a mitigação de vulnerabilidades. O panorama formado por esses três pontos de vista corresponde ao contexto em que as contribuições deste trabalho estão inseridas.

A seção 3, Proposta, descreve um método para o desenvolvimento de componentes de HTML que mantenham invisíveis, para o restante da página, as informações mantidas ou geradas por esses componentes, ao mesmo tempo em que expõe uma interface de programação baseada em controle do acesso à informação encapsulada. São apresentadas nesta seção a disponibilidade dos recursos necessários para a implementação do método, bem como suas limitações de uso.

Na seção 4, Avaliação, são propostos critérios para a verificação da eficácia do método proposto: disponibilidade nas plataformas de navegação, limites de proteção versus vulnerabilidades mitigadas, e requisitos de funcionamento. A seção se completa com a aplicação desses critérios sobre o método proposto, em comparação com trabalhos embasados pela abordagem de IFC -- o controle de fluxo de informação define, no âmbito do problema, maior granularidade na segurança da informação em Javascript, ao custo da compatibilidade com a base instalada de navegadores.

O conteúdo da seção 5, Conclusões, deriva da reflexão crítica sobre a implementação do método proposto em contraponto aos resultados observados na avaliação qualitativa. Recomendações sobre a aplicação do método, além de oportunidades a serem exploradas por trabalhos futuros, fecham a conclusão dos esforços deste trabalho.


% ----------------------------------------------------------
% ELEMENTOS PÓS-TEXTUAIS
% ----------------------------------------------------------
\postextual
%!TEX root = Projeto.tex
% ----------------------------------------------------------
% Referências bibliográficas
% ----------------------------------------------------------
\setboolean{isBib}{true} % Obriga o titulo das referências à esquerda

\setlength\bibitemsep{12pt} % Espaço de uma linha entre as referencias

\bibliography{Projeto}

\setboolean{isBib}{false} % Desobriga o titulo das referências à esquerda

% ----------------------------------------------------------
% Apêndices
% ----------------------------------------------------------

%\apendices
%\partapendices
%\chapter{Lorem ipsum dolor sit amet}

%\lipsum[21-26]

% ----------------------------------------------------------
% Anexos
% ----------------------------------------------------------

% \anexos
% \partanexos
% \chapter{Padrões de projeto para mitigação de riscos}

\section{}
%\lipsum[26-32]




\end{document}
