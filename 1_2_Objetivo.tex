%!TEX root = Projeto.tex
\section{Objetivo}

%A MOTIVAÇÃO DEVE TER ESCLARECIDO QUE EXISTEM PROBLEMAS DE INFOSEC DIFÍCEIS DE SE CONTORNAR COM AS DIRETIVAS TRADICIONAIS. NESTA SEÇÃO DEVERÁ FICAR CLARO QUE UMA ABORDAGEM DIFERENTE ""PODE"" SER AVALIADA, E QUE É OBJETIVO DESTE TRABALHO FAZER ESSA AVALIAÇÃO.

%Para formular um Objetivo realista, é preciso ter uma hipótese (hipo-tese).
%Hipótese: uma afirmação que deverá ser testada (parece razoável, baseando-me no que li, ou no que observei, que é possível fazer X...). É preciso justificar a hipótese.

O objetivo deste trabalho é propor uma abordagem para o confinamento de informação usando recursos padronizados de HTML e Javascript. A proposta deverá permitir ao desenvolvedor a delimitação de regiões do DOM cujo conteúdo seja opaco para scripts incorporados e extensões do navegador.

A efetividade da proposta deve ser avaliada segundo requisitos não funcionais enumerados por \cite{DeRyck2012}:

\begin{alineas}
	\item Efetividade da separação do DOM: a estrutura de documento mantida em regiões ocultas pela abordagem proposta é separada do restante da página;
	\item Efetividade do isolamento de scripts: scripts carregados em regiões ocultas não poderão sofrer influência de scripts externos a essas regiões;
	\item Confidencialidade: a informação mantida em regiões ocultas só poderá ser lida por scripts especialmente criados para esse fim pelo desenvolvedor da aplicação;
	\item Integridade: a informação mantida em regiões ocultas só poderá ser modificada por scripts especialmente criados para esse fim pelo desenvolvedor da aplicação;
	\item Autenticidade: o protocolo de comunicação com regiões ocultas deve suportar apenas participantes que confiem uns aos outros.
\end{alineas}

Requisitos funcionais da proposta devem satisfazer sua compatibilidade com os navegadores modernos, não-experimentais:

\begin{alineas}
	\item Permitir que qualquer combinação de elementos HTML seja encapsulada em regiões ocultas;
	\item Ser compatível com bibliotecas e \textit{frameworks} de desenvolvimento em Javascript, HTML e CSS;
	\item Expor uma interface de programação para a leitura e modificação das informações contidas em regiões ocultas.
\end{alineas}

Como objetivo secundário, será implementada uma prova de conceito que valide os requisitos estabelecidos.