%!TEX root = Projeto.tex
\section{Objetivo}

%A MOTIVAÇÃO DEVE TER ESCLARECIDO QUE EXISTEM PROBLEMAS DE INFOSEC DIFÍCEIS DE SE CONTORNAR COM AS DIRETIVAS TRADICIONAIS. NESTA SEÇÃO DEVERÁ FICAR CLARO QUE UMA ABORDAGEM DIFERENTE ""PODE"" SER AVALIADA, E QUE É OBJETIVO DESTE TRABALHO FAZER ESSA AVALIAÇÃO.

%Para formular um Objetivo realista, é preciso ter uma hipótese (hipo-tese).
%Hipótese: uma afirmação que deverá ser testada (parece razoável, baseando-me no que li, ou no que observei, que é possível fazer X...). É preciso justificar a hipótese.

O objetivo deste trabalho é propor uma abordagem para o confinamento de informação usando recursos padronizados de HTML e Javascript. A proposta deverá permitir ao desenvolvedor a delimitação de regiões do DOM cujo conteúdo seja opaco para scripts incorporados e extensões do navegador.

Como objetivo secundário, será implementada uma prova de conceito. Nela a proposta será avaliada sob dois requisitos: (1) efetividade da ocultação do conteúdo de regiões do DOM, e (2) efetividade da ocultação dos eventos do DOM originado em regiões do DOM.

Para a implementação serão empregadas APIs suportadas pelos navegadores modernos, sem que seja necessário fazer uso de técnicas que exijam navegadores experimentais.