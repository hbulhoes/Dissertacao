%!TEX root = Projeto.tex
\section{Objetivo}

%A MOTIVAÇÃO DEVE TER ESCLARECIDO QUE EXISTEM PROBLEMAS DE INFOSEC DIFÍCEIS DE SE CONTORNAR COM AS DIRETIVAS TRADICIONAIS. NESTA SEÇÃO DEVERÁ FICAR CLARO QUE UMA ABORDAGEM DIFERENTE ""PODE"" SER AVALIADA, E QUE É OBJETIVO DESTE TRABALHO FAZER ESSA AVALIAÇÃO.

%Para formular um Objetivo realista, é preciso ter uma hipótese (hipo-tese).
%Hipótese: uma afirmação que deverá ser testada (parece razoável, baseando-me no que li, ou no que observei, que é possível fazer X...). É preciso justificar a hipótese.

Buscando alternativas de desenvolvimento defensivo que não se baseiem em implementações experimentais, este trabalho investiga um método para o confinamento de informação através de recursos padronizados de HTML e Javascript. Interfaces de programação (APIs -- \textit{application programming interface}) suportadas (ou em vias de adoção) pelos navegadores modernos, como o \textit{shadow DOM} \cite{W3C:ShadowDOM} e o elemento \textit{<iframe>} \cite{MDN:iframes}, dispõem de recursos que possibilitariam o encapsulamento de informações em componentes de estrutura opaca, seja para \textit{scripts} incorporados na página HTML como para \textit{plugins} do navegador. Hipoteticamente, é possível propor um método que, combinando tais recursos de APIs, ofereça ao desenvolvedor uma ferramenta para a composição de páginas imunes a ataques de vazamento de informação. Implementando o método e colocando-o à prova por meio de testes que simulam esse tipo de ataque, o objetivo deste trabalho é avaliar a efetividade do método como meio de ofuscação (1) das informações incorporadas e exibidas na página e (2) do comportamento do usuário ao interagir com componentes HTML e Javascript desenvolvidos conforme o método proposto.