%!TEX root = Projeto.tex
\imprimircapa
\imprimirfolhadeaprovacao
\imprimirfolhaderosto

% ---
% Inserir a ficha bibliografica
% ---

% Isto é um exemplo de Ficha Catalográfica, ou ``Dados internacionais de
% catalogação-na-publicação''. Você pode utilizar este modelo como referência.
% Porém, provavelmente a biblioteca da sua universidade lhe fornecerá um PDF
% com a ficha catalográfica definitiva após a defesa do trabalho. Quando estiver
% com o documento, salve-o como PDF no diretório do seu projeto e substitua todo
% o conteúdo de implementação deste arquivo pelo comando abaixo:
%
% \begin{fichacatalografica}
%     \includepdf{fig_ficha_catalografica.pdf}
% \end{fichacatalografica}
\begin{fichacatalografica}
	\vspace*{\fill}					% Posição vertical
	\hrule							% Linha horizontal
	\begin{center}					% Minipage Centralizado
		\begin{minipage}[c]{12.5cm}		% Largura
			\setlength{\parskip}{1em}

			\imprimirautor
		
			\hspace{0.5cm} \imprimirtitulo~/~\imprimirautor. \imprimirlocal, \imprimirdata.

			\hspace{0.5cm} \pageref{LastPage} p. : il. (algumas color.); 30 cm.
		
			\hspace{0.5cm} \imprimirorientadorRotulo~\imprimirorientador
		
			\hspace{0.5cm} \parbox[t]{\textwidth}{\imprimirtipotrabalho~--~\imprimirinstituicao, \imprimirdata.}
		
			\hspace{0.5cm}
		%		1. Palavra-chave1.
		%		2. Palavra-chave2.
		%		I. Orientador.
		%		II. Universidade xxx.
		%		III. Faculdade de xxx.
		%		IV. Título\\
		
			\hspace{8.75cm} CDU XX:XXX:XXX.X\\
	
		\end{minipage}
	\end{center}
	\hrule
\end{fichacatalografica}
% ---



% ---
% Dedicatória
% ---
%\begin{dedicatoria}
%\vspace*{\fill}
%\OnehalfSpacing
%%\centering
%%\noindent
%Dedico este trabalho...
%
%
%\vspace*{\fill}
%\end{dedicatoria}
% ---

% ---
% Agradecimentos
% ---
%\begin{agradecimentos}
%\vspace*{\fill}
%\OnehalfSpacing
%Gostaria de agradecer...
%
%
%\vspace*{\fill}
%\end{agradecimentos}
% ---

% ---
% Epígrafe
% ---
%\begin{epigrafe}
%    \vspace*{\fill}
%	\begin{flushright}
%		\textit{``Não vos amoldeis às estruturas deste mundo, \\
%		mas transformai-vos pela renovação da mente, \\
%		a fim de distinguir qual é a vontade de Deus: \\
%		o que é bom, o que Lhe é agradável, o que é perfeito.\\
%		(Bíblia Sagrada, Romanos 12, 2)}
%	\end{flushright}
%\end{epigrafe}
% ---


%!TEX root = Projeto.tex
\newpage
\begin{resumo}
\normalsize

% Breve contextualização do problema
Os programas navegadores oferecem amplos recursos para a execução de aplicações voltadas para a web. Suas funcionalidades, porém, podem expor a informação dos usuários a riscos de vazamento e adulteração. Ainda que os navegadores venham sendo melhorados para que consigam detectar e mitigar alguns desses riscos, esse esforço é marcado por concessões às capacidades esperadas pelas aplicações web, fazendo com que a segurança da informação no navegador seja um campo de conhecimento com iniciativas e padrões consideradas incoerentes entre si. Por causa disso, desenvolvedores de aplicações nem sempre podem prever o grau de exposição dos dados de seus usuários.
% Dica do que pode ser feito para minimizar o problema
Seria importante que fosse possível determinar, de modo programável e imperativo, que esses dados ficassem fora do alcance de participantes não confiáveis, particularmente \scripts.
% Proposta do trabalho
Para essa finalidade, este trabalho propõe uma abordagem que proporcione ao desenvolvedor uma barreira de proteção incorporada às aplicações. Nesta proposta, o desenvolvedor tem controle direto sobre a exposição das informações que considerar sensíveis, em contraste com o controle indireto, declarativo e tolerante a vazamento de informação empregado atualmente nos navegadores.
% Como será validado
A validação dessa proposta ocorrerá por meio de um protótipo de sistema submetido a situações de risco de vazamento da informação em páginas web, em correspondência com ocorrências documentadas pela literatura. O protótipo será preparado para que a abordagem proposta, sob a forma de um \script baseado em APIs padronizadas de HTML e Javascript, seja colocada à prova no seu propósito de neutralizar tais situações de risco.


\vspace{\onelineskip}

\noindent
\textbf{Palavras-chave:} \imprimirpalavraschave
\end{resumo}

% resumo em inglês
%\begin{resumo}[Abstract]
%%\begin{otherlanguage*}{english}
%Resumo da dissertação em inglês.
%
%
%\vspace{\onelineskip}
%
%\noindent
%\textbf{Keywords:} \imprimirkeyword   
%%\end{otherlanguage*}
%\end{resumo}

% ---
% inserir lista de ilustrações
% ---
{\SingleSpacing
\pdfbookmark[0]{\listfigurename}{lof}
\listoffigures*
\cleardoublepage
}
% ---

% ---
% inserir lista de trechos de código fonte
% ---
{\SingleSpacing
	\pdfbookmark[0]{\lstlistlistingname}{lol}
	\lstlistoflistings
	\cleardoublepage
}
% ---

% ---
% inserir lista de quadros
% ---
%\pdfbookmark[0]{\listofquadrosname}{loq}
%\listofquadros*
%\cleardoublepage
% ---

% ---
% inserir lista de tabelas
% ---
%{\SingleSpacing
%\pdfbookmark[0]{\listtablename}{lot}
%\listoftables*
%\cleardoublepage
%}
% ---

% ---
% inserir lista de abreviaturas e siglas
% ---
%
\begin{siglas}
  \item[CORS] Cross-Origin Resource Sharing
  \item[CSRF] Cross-Site Request Forgery
  \item[CSP] Content Security Policy
  \item[DAC] Discretionary Access Control
  \item[DOM] Document Object Model
  \item[IFC] Information Flow Control
  \item[JSON] JavaScript Object Notation
  \item[MAC] Mandatory Access Control
  \item[SOP] Same Origin Policy
  \item[SRI] Subresource Integrity
  \item[XSS] Cross-Site Scripting
\end{siglas}

% ---

% ---
% inserir lista de símbolos
% ---
%\begin{simbolos}
%  \item[$ \Gamma $] Letra grega Gama
%  \item[$ \Lambda $] Lambda
%  \item[$ \zeta $] Letra grega minúscula zeta
%  \item[$ \in $] Pertence
%\end{simbolos}
% ---

% ---
% inserir o sumario
% ---
{\SingleSpacing
	\cleardoublepage
	\pdfbookmark[0]{\contentsname}{toc}
	\tableofcontents*
	\cleardoublepage
}
% ---