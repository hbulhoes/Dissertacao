%!TEX root = Projeto.tex
\imprimircapa
\imprimirfolhadeaprovacao
\imprimirfolhaderosto

% ---
% Inserir a ficha bibliografica
% ---

% Isto é um exemplo de Ficha Catalográfica, ou ``Dados internacionais de
% catalogação-na-publicação''. Você pode utilizar este modelo como referência.
% Porém, provavelmente a biblioteca da sua universidade lhe fornecerá um PDF
% com a ficha catalográfica definitiva após a defesa do trabalho. Quando estiver
% com o documento, salve-o como PDF no diretório do seu projeto e substitua todo
% o conteúdo de implementação deste arquivo pelo comando abaixo:
%
% \begin{fichacatalografica}
%     \includepdf{fig_ficha_catalografica.pdf}
% \end{fichacatalografica}
\begin{fichacatalografica}
	\vspace*{\fill}					% Posição vertical
	\hrule							% Linha horizontal
	\begin{center}					% Minipage Centralizado
		\begin{minipage}[c]{12.5cm}		% Largura
			\setlength{\parskip}{1em}

			\imprimirautor
		
			\hspace{0.5cm} \imprimirtitulo~/~\imprimirautor. \imprimirlocal, \imprimirdata.

			\hspace{0.5cm} \pageref{LastPage} p. : il. (algumas color.); 30 cm.
		
			\hspace{0.5cm} \imprimirorientadorRotulo~\imprimirorientador
		
			\hspace{0.5cm} \parbox[t]{\textwidth}{\imprimirtipotrabalho~--~\imprimirinstituicao, \imprimirdata.}
		
			\hspace{0.5cm}
		%		1. Palavra-chave1.
		%		2. Palavra-chave2.
		%		I. Orientador.
		%		II. Universidade xxx.
		%		III. Faculdade de xxx.
		%		IV. Título\\
		
			\hspace{8.75cm} CDU XX:XXX:XXX.X\\
	
		\end{minipage}
	\end{center}
	\hrule
\end{fichacatalografica}
% ---



% ---
% Dedicatória
% ---
%\begin{dedicatoria}
%\vspace*{\fill}
%\OnehalfSpacing
%%\centering
%%\noindent
%Dedico este trabalho...
%
%
%\vspace*{\fill}
%\end{dedicatoria}
% ---

% ---
% Agradecimentos
% ---
%\begin{agradecimentos}
%\vspace*{\fill}
%\OnehalfSpacing
%Gostaria de agradecer...
%
%
%\vspace*{\fill}
%\end{agradecimentos}
% ---

% ---
% Epígrafe
% ---
%\begin{epigrafe}
%    \vspace*{\fill}
%	\begin{flushright}
%		\textit{``Não vos amoldeis às estruturas deste mundo, \\
%		mas transformai-vos pela renovação da mente, \\
%		a fim de distinguir qual é a vontade de Deus: \\
%		o que é bom, o que Lhe é agradável, o que é perfeito.\\
%		(Bíblia Sagrada, Romanos 12, 2)}
%	\end{flushright}
%\end{epigrafe}
% ---


%!TEX root = Projeto.tex
\newpage
\begin{resumo}
\normalsize

Impulsionados por recursos avançados de HTML e Javascript, os programas navegadores da web tornaram-se plataformas de desenvolvimento e publicação de aplicativos. O aumento em escopo de funcionalidade dos navegadores, porém, vem acompanhado de vulnerabilidades de segurança da informação que podem expor dados do usuário. Muitas dessas vulnerabilidades se manifestam de formas sutis, mantendo-se fora do alcance das políticas de segurança padronizadas e adotadas pela indústria, e afetando a confidencialidade da informação que trafega pelo navegador. Para neutralizar esse problema diversas iniciativas experimentais têm sido propostas, sem que tenham encontrado, até o momento, adoção em massa. Ao mesmo tempo, novos recursos de programação têm sido incorporados aos navegadores, e alguns deles, quando utilizados em conjunto, levantam a possibilidade de implementação do encapsulamento da informação, tornando-a invisível a agentes não autorizados. O objetivo deste trabalho é propor um método que, fundamentado em capacidades de programação padronizadas e difundidas, implemente o encapsulamento da informação em HTML e Javascript, o que poderia tornar realidade alguns dos requisitos de segurança que apenas abordagens experimentais têm conseguido assegurar. Para tanto, este trabalho apresentará uma especificação desse método, uma implementação modelo e uma avaliação da eficácia do método em relação a determinados casos de teste que evidenciam problemas de confidencialidade da informação.

\vspace{\onelineskip}

\noindent
\textbf{Palavras-chave:} \imprimirpalavraschave
\end{resumo}

% resumo em inglês
%\begin{resumo}[Abstract]
%%\begin{otherlanguage*}{english}
%Resumo da dissertação em inglês.
%
%
%\vspace{\onelineskip}
%
%\noindent
%\textbf{Keywords:} \imprimirkeyword   
%%\end{otherlanguage*}
%\end{resumo}

% ---
% inserir lista de ilustrações
% ---
{\SingleSpacing
\pdfbookmark[0]{\listfigurename}{lof}
\listoffigures*
\cleardoublepage
}
% ---

% ---
% inserir lista de trechos de código fonte
% ---
{\SingleSpacing
	\pdfbookmark[0]{\lstlistlistingname}{lol}
	\lstlistoflistings
	\cleardoublepage
}
% ---

% ---
% inserir lista de quadros
% ---
%\pdfbookmark[0]{\listofquadrosname}{loq}
%\listofquadros*
%\cleardoublepage
% ---

% ---
% inserir lista de tabelas
% ---
{\SingleSpacing
	\pdfbookmark[0]{\listtablename}{lot}
	\listoftables*
	\cleardoublepage
}
% ---

% ---
% inserir lista de abreviaturas e siglas
% ---
%
\begin{siglas}
  \item[CORS] Cross-Origin Resource Sharing
  \item[CSRF] Cross-Site Request Forgery
  \item[CSP] Content Security Policy
  \item[DAC] Discretionary Access Control
  \item[DOM] Document Object Model
  \item[IFC] Information Flow Control
  \item[JSON] JavaScript Object Notation
  \item[MAC] Mandatory Access Control
  \item[SOP] Same Origin Policy
  \item[SRI] Subresource Integrity
  \item[XSS] Cross-Site Scripting
\end{siglas}

% ---

% ---
% inserir lista de símbolos
% ---
%\begin{simbolos}
%  \item[$ \Gamma $] Letra grega Gama
%  \item[$ \Lambda $] Lambda
%  \item[$ \zeta $] Letra grega minúscula zeta
%  \item[$ \in $] Pertence
%\end{simbolos}
% ---

% ---
% inserir o sumario
% ---
{\SingleSpacing
	\cleardoublepage
	\pdfbookmark[0]{\contentsname}{toc}
	\tableofcontents*
	\cleardoublepage
}
% ---