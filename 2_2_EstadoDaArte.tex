%!TEX root = Projeto.tex

%VALE A PENA OLHAR O QUE PODE SER APROVEITADO DOS TRABALHOS ENUMERADOS COMO "SANDBOXING" NO ARTIGO SOBRE COWL.

\section{Estado da arte}
%O estado da arte em segurança da informação nos navegadores é resultado de uma progressiva adaptação deste tipo de software aos recursos de composição e de publicação das aplicações web. 


%é uma preocupação desde o início da exploração comercial da web: o protocolo HTTPS, por exemplo, foi suportado pela primeira vez em 1994 pelo navegador Netscape. Mas é a partir de 1996, com a introdução da linguagem Javascript e de APIs para programação nos navegadores, que as vulnerabilidades de segurança da informação no \textit{front-end} ganharam relevância. A existência de \textit{conteúdo ativo} no navegador inaugurou um legado de preocupações sobre a confiabilidade, as finalidades e os privilégios de execução de scripts, e também dos mecanismos de segurança que ao longo do tempo foram incorporados pela indústria.







