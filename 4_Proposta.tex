% !TeX spellcheck = <none>
%!TEX root = Projeto.tex
\chapter{Proposta}

Esta dissertação propõe um mecanismo discricionário contra vazamento de informação contida no DOM baseado em duas capacidades: (1) impedir que {\scripts} encontrem nós específicos do DOM via APIs de manipulação de árvore como |document.getElementById()|, |ParentNode.querySelector()| e |Element.childNodes| e (2) impedir que {\scripts} sejam notificados da ocorrência de eventos do DOM originários de nós específicos. Entretanto, para que a informação derivada dessas funcionalidades possa ser útil aos autores de aplicações web, o mecanismo proposto oferece uma API proprietária que atua como \poe{proxy} ante os nós ocultados.%, \dubious{caracterizável pelos aspectos X, Y e Z das políticas discricionárias de segurança da informação}.

As próximas seções encarregam-se de apresentar o mecanismo em termos de sua especificação e de seu funcionamento, descrevendo as tecnologias em que ele é fundamentado e sua aptidão frente aos casos de uso levantados na \dubious{seção X}.

% !TeX spellcheck = <none>
%!TEX root = Projeto.tex

\section{API de ocultação do DOM fundamentada em Shadow DOM}

Com base no isolamento de árvores de DOM proporcionado pelo recurso de \poe{shadow DOM}, descrito preliminarmente em \ref{Section: introShadowDOM}, é possível explorar o comportamento derivado de uma propriedade das APIs que estabelece, em sua especificação, que para todo \poe{shadow host} será determinada a visibilidade de sua \poe{shadow tree}. O conceito de \textit{visibilidade}, neste contexto, corresponde à possibilidade de que {\scripts} consigam alcançar nós do DOM contidos na \poe{shadow tree}, tanto indiretamente, enumerando os nós-filhos da árvore por meio da propriedade |ShadowRoot.childNodes|, quanto diretamente, por meio de métodos como |getElementById()| ou |querySelector()|, implementados por objetos como |Element|, |HTMLDocument| e |DocumentFragment|.

O atributo de visibilidade é determinado no momento da criação do \poe{shadow host}. A listagem de código \ref{Src: jsShadowHostCtor}, entre as linhas 2 e 6, exemplifica os efeitos mais evidentes desse atributo. Essencialmente, os {\scripts} ganham ou perdem acesso à propriedade |Element.shadowRoot|, na medida em que o parâmetro |mode| é determinado como |open| ou |closed|.

\lstinputlisting[language=JavaScript,
	inputencoding=utf8,
	label={Src: jsShadowHostCtor},
	caption={Determinação da visibilidade da \poe{shadow tree} e sua consequência na tentativa de acesso ao DOM subjacente}]
	{codigo/sample04-shadow-host-ctor.js}

Tornando inacessível a propriedade |Element.shadowRoot|, o método |Element.attachShadow({mode:'closed'})| provê um meio programático para a definição de regiões protegidas na estrutura DOM da página web. No entanto, a inacessibilidade do conteúdo da \poe{shadow root} depende de medidas que vão além das garantias previstas pela especificação da API de Shadow DOM. Isso se deve à extensibilidade da linguagem Javascript, que permite que {\scripts} interceptem quaisquer métodos de qualquer API. Desse modo, um {\script} cuidadosamente escrito poderia ganhar acesso a todas as \poe{shadow roots} criadas em uma página web, independentemente da opção |open| ou |closed| atribuída ao parâmetro |mode| em |attachShadow()|. Essa estratégia é detalhada na seção seguinte. %exemplificada na listagem de código \ref{Src: jsShadowApiIntercept}.

\subsection{Interceptação de métodos nativos do ambiente de execução}

A linguagem Javascript apresenta \textit{herança prototípica} como mecanismo para compartilhar estrutura, comportamento e estado entre os objetos que participam de um mesmo relacionamento denominado de \textit{cadeia prototípica}. Por esse mecanismo, a linguagem Javascript admite o uso de abstrações de classes, a exemplo de linguagens estáticas como Java e C++ \cite[seção 4.2.1]{ECMA2017}. Porém, ao contrário dessas linguagens, Javascript possibilita a redefinição dinâmica dos membros dos protótipos/classes. A listagem \ref{Src: jsPrototypicalInheritance} exemplifica esse mecanismo.

%Objetos em Javascript são entidades de estrutura dinâmica, comportando-se como dicionários ou \poe{hashes} de informação. Por meio de herança prototípica, objetos em Javascript são definidos como uma composição formada tanto pelos atributos herdados da cadeia de protótipos quanto dos atributos associados diretamente ao objeto.

\lstinputlisting[language=JavaScript,
	inputencoding=utf8,
	label={Src: jsPrototypicalInheritance},
	caption={Demonstração de herança prototípica em Javascript}]
	{codigo/sample05-prototype.js}

Uma consequência da flexibilidade oferecida pela herança prototípica é que um {\script} pode modificar os métodos de um protótipo de forma que o acionamento desses métodos seja substituído ou encapsulado por código arbitrário. Essa capacidade de interceptação permite que métodos sejam instrumentados, monitorados ou mesmo corrompidos sem que o \poe{runtime} da linguagem levante impedimentos. O exemplo na listagem \ref{Src: jsMethodHijacking} demonstra como o método |Number.toString()| pode ser desvirtuado para retornar informação incorreta -- neste caso, o método sempre retornará a mesma \poe{string}, ``42''.

\lstinputlisting[language=JavaScript,
	inputencoding=utf8,
	label={Src: jsMethodHijacking},
	caption={Exemplo de redefinição do método toString() do objeto Number}]
	{codigo/sample06-hijack.js}

Virtualmente qualquer dos membros de objetos em Javascript, incluindo propriedades e métodos dos objetos primitivos |Boolean|, |Number|, |String| e |Object|, podem ser redefinidos em suas cadeias de herança. \citeinline{Magazinius2012}, em seu estudo sobre a efetividade das estratégias empregadas por \poe{sandboxes} como Google Caja e Facebook Javascript, denomina a redefinição mal-intencionada da cadeia de herança como ``envenenamento do protótipo'' (\poe{prototype poisoning}, no artigo original). É por esse motivo que a mera utilização de |{mode: 'closed'}| como parâmetro do método |Element.attachShadow()| não é suficiente para tornar invioláveis as \poe{shadow roots} que forem criadas com a intenção de serem ``ocultas'', pois a redefinição desse método, exemplificada pela listagem \ref{Src: jsShadowHijacking}, tem acesso todas as referências às \poe{shadow roots} criadas durante o ciclo de vida de uma página, podendo ler e modificar o conteúdo tido como protegido pelo desenvolvedor.

\lstinputlisting[language=JavaScript,
	inputencoding=utf8,
	label={Src: jsShadowHijacking},
	caption={Exemplo de interceptação do método Element.attachShadow()}]
	{codigo/sample07-attachShadowHijack.js}

\subsubsection{Uma estratégia para neutralizar a redefinição de APIs}

No \poe{runtime} de Javascript, a ordem de declaração dos {\scripts} determina sua sequência de execução, a qual ocorre em série -- é garantido que apenas um {\script} seja executado em um dado momento \cite[seção 8.3]{ECMA2017}. Esse comportamento sustenta a elaboração de uma estratégia para mitigação dos riscos inerentes à interceptação dinâmica dos métodos em Javascript: o primeiro {\script} a ser executado poderia armazenar referências aos métodos que lhe forem importantes, em uma rotina de \poe{setup}. Desta forma, mesmo se os {\scripts} carregados subsequentemente efetuarem sobrescrita de protótipos, as implementações originais dos métodos requeridos estarão a salvo da manipulação. Para tanto, as seguintes garantias precisam ser obedecidas:

\begin{alineas}
	\item Garantia de que um {\script} confiável de \poe{setup} seja o primeiro bloco de código Javascript carregado pelo navegador na página web.
	\item Garantia de que os métodos de que o desenvolvedor necessite para interação com informações sensíveis sejam referenciados por variáveis locais do {\script} de \poe{setup}.
	\item Garantia de que as chamadas aos métodos referenciados ocorram por via indireta, ignorando os |prototypes| de que fizerem parte.
\end{alineas}

A terceira garantia depende da invocação de funções por meio do método |Function.apply(thisObj, methodFn)| em detrimento de chamadas diretas como |thisObj.methodFn()|. A listagem \ref{Src: jsMethodProtection} fornece um exemplo de implementação dessa estratégia.

\lstinputlisting[language=JavaScript,
	inputencoding=utf8,
	label={Src: jsMethodProtection},
	caption={Implementação da estratégia de chamada indireta de métodos}]
	{codigo/sample08-method-protection-strategy.js}


\subsubsection{Falhas na estratégia de neutralização}
Ainda que o algoritmo de neutralização apresente certa capacidade de neutralizar os riscos de envenenamento do protótipo de objetos, sua implementação está sujeita a outros riscos. Em primeiro lugar, a garantia de que determinado {\script} seja carregado antes dos demais é, na prática, frágil. Os {\scripts} interessados nessa proteção podem não estar sob controle do autor de uma aplicação web, especialmente se fizerem parte de soluções de terceiros cuja ordem de carregamento e execução seja imprevisível \poe{a priori}. Ademais, em segundo lugar, é plenamente possível que um {\script} mal-intencionado tome controle exatamente do método |Function.apply()|, o que tornaria vulneráveis todas as chamadas de função dependentes desse método. A listagem \ref{Src: jsApplySubversion} evidencia essa modalidade de interceptação, demonstrando a relativa simplicidade de se contornar a estratégia proposta.

\lstinputlisting[language=JavaScript,
	inputencoding=utf8,
	label={Src: jsApplySubversion},
	caption={Subversão do método Function.apply() como meio de ganhar acesso às \poe{shadow roots} fechadas}]
	{codigo/sample09-apply-method-subversion.js}

Pelas fragilidades apontadas, uma outra estratégia deve ser proposta -- uma que seja mais tolerante quanto às garantias necessárias, porém mais robusta em ambientes de execução hostis.


\subsection{Detectando a interceptação de métodos em Javascript}

Em vista das características dinâmicas da linguagem Javascript, e em consequência à tolerância que o navegador demonstra para com todos os {\scripts} incorporados em uma página web, este trabalho toma como premissa que não é factível estabelecer um ambiente de execução imune às diversas maneiras nas quais as APIs de Javascript podem ser exploitadas, especialmente quando o objetivo do agente mal-intencionado é facilitar o vazamento de informação. Ao contrário, este trabalho propõe uma abordagem que pode contribuir para que as aplicação web não sejam expostas aos riscos derivados de {\scripts} que interfiram na cadeia de herança prototípica. Por meio dessa abordagem, a aplicação determina, tão cedo quanto possível no ciclo de vida da página, se as APIs de que depende foram de algum modo comprometidas e, em caso positivo, que nenhuma informação confidencial seja exposta. Esta é uma estratégia de \poe{early fail}: se determinadas condições não forem atendidas na inicialização da página, que nenhuma operação posterior tenha a oportunidade de ser executada.

Para tanto, é empregado um dispositivo de software que detecta a ocorrência de interceptação de métodos em Javascript. Esse dispositivo corresponde a um processo capaz de avaliar, em tempo de inicialização, se determinados métodos de API tiveram sido substituídos por interceptadores, o que para os propósitos do dispositivo os caracteriza como não-confiáveis. A existência de ao menos um método não-confiável é suficiente para que o dispositivo acuse uma falha de inicialização, a partir do que a aplicação web poderá desviar o fluxo para uma condição de erro. A não-ocorrência de falha, por sua vez, significa que os métodos avaliados não foram interceptados, sendo considerados como confiáveis. O diagrama 3.1.2 ilustra a sequência de atividades desempenhadas durante o processo de detecção de métodos não-confiáveis.

[DIAGRAMA 3.1.2]

A implementação do dispositivo de detecção se baseia na conversão implícita de valores passados como argumento para os métodos avaliados. O dispositivo é capaz de detectar interceptação em métodos que esperam ao menos um parâmetro do tipo |String| e que, para tais parâmetros que receberem como argumentos valores de outros tipos, o método tente convertê-los em \poe{strings}. Em Javascript, todo objeto herda do protótipo |Object| o método |toString()| \cite[seção 19.1]{ECMA2017}, cujo propósito é criar uma representação textual do valor intrínseco de um objeto. Embora esse propósito seja distinto de uma ``conversão de tipo'', este é o efeito observado em situações como a do seguinte trecho de código:

\begin{lstlisting}[language=javascript]
var	o = { toString: function () { return 'Hello, world'; } };
document.write(o);
// A expressão "Hello, world" terá sito escrita na página web
\end{lstlisting}

Implicitamente o método |document.write()| ``converte'' o argumento |o| em um valor |String| por meio da chamada de seu método |toString()| (o qual, de fato, sobrescreve o método |toSring()| herdado de |Object| através da cadeia prototípica). Esse padrão de conversão é empregado em outras APIs de Javascript e também em operações como a de concatenação de \poe{strings}:

\begin{lstlisting}[language=javascript]
var	s = o + ' - have a nice day!';
// O conteúdo da variável s é "Hello, world - have a nice day"
\end{lstlisting}

Para fins do dispositivo de detecção de interceptação, um objeto não-textual (denominado \textit{vetor}) é passado como argumento para os métodos sob avaliação (denominados \textit{suspeitos}) e estes, ao realizarem a conversão do vetor em |String|, irão implicitamente acionar o método |vetor.toString()|. Mais do que devolver uma representação textual do vetor, a implementação de |toString()| tem a responsabilidade de obter informações sobre o contexto de execução do método suspeito e, a partir disso, inferir se esse contexto pertence a uma interceptação de API. Essa inferência se baseia na profundidade da pilha de chamadas (\poe{call stack}) em termos do número de chamadas aninhadas (\poe{stack frames}). O dispositivo emprega uma heurística cuja premissa é de que deve existir uma número predeterminado de \poe{stack frames} entre a chamada de |vetor.toString()| e a chamada do próprio método suspeito. Os experimentos demonstram que a profundidade de pilha esperada para todos os métodos avaliados é 1: um número maior de \poe{frames} significa que há mais chamadas em curso, caracterizando a interceptação do método suspeito. A listagem \ref{Src: jsDetectionVector} demonstra a implementação de vetor utilizada nos experimentos deste trabalho.

\lstinputlisting[language=JavaScript,
	inputencoding=utf8,
	label={Src: jsDetectionVector},
	caption={Implementação do vetor de detecção de métodos interceptados}]
	{codigo/sample10-detection-vector.js}


O método |vetor.toString()|, núcleo do dispositivo de detecção, avalia a profundidade da pilha de chamadas pela inspeção da propriedade |stack| de um objeto |Error|. Embora não seja padronizada, a propriedade |Error.stack| apresenta comportamento consistente entre os mecanismos de execução de Javascript, retornando uma sequência de linhas de texto correspondentes às \poe{stack frames} -- um efeito demonstrado na listagem \ref{Src: jsCallStackSampler}. Assim, a contagem de linhas de texto contidas no \poe{stack trace} é o resultado tomado como profundidade da pilha.

Experimentos evidenciam que, em diferentes plataformas\footnote{NodeJS v8.9.4; Chrome 66; Microsoft Edge estão em conformidade entre si; notável exceção do Firefox Quantum 59.}, não é possível substituir a implementação nativa da propriedade |Error.stack|. Neste trabalho, foram desenvolvidos {\scripts} que tentaram, sem sucesso, interceptar e modificar o conteúdo de |Error.prototype.stack|, além da substituição do próprio construtor e protótipo do objeto |Error|. Independentemente dessas tentativas, os \poe{runtimes} de Javascript reforçam uma característica de previsibilidade no comportamento do objeto |Error|. Essa constatação dá segurança para que o vetor utilizado pelo dispositivo seja considerado confiável em sua capacidade de detectar variações na pilha de chamadas.

\lstinputlisting[language=JavaScript,
	inputencoding=utf8,
	label={Src: jsCallStackSampler},
	caption={Diferentes conteúdos de Error.stack quando se consideram aninhamentos mais profundos de chamadas de função}]
	{codigo/sample11-call-stack-sampler.js}
\input{4_2_CasosDeTeste}
\input{4_3_ApresentacaoJstestMe}

\section{Roteiro para a elaboração da proposta}

\begin{alineas}
	\item \textbf{Mapeamento de cenários.} Partindo dos requisitos não funcionais estabelecidos, serão definidos cenários de testes capazes de evidenciar as condições necessárias para a não-conformidade com esses requisitos. Os cenários assim definidos delimitarão o escopo de ação da abordagem proposta por este trabalho.
	\item \textbf{Especificação da proposta.} A abordagem derivada dos cenários de testes será formalmente especificada, em termos das interfaces e funcionalidades esperadas, visando orientar o desenvolvimento de uma implementação de teste e provas de conceito.
	\item \textbf{Implementação do mecanismo de ocultação.} A materialização da abordagem proposta será efetuada pelo desenvolvimento dos scripts necessários para a implementação das interfaces esperadas, em conformidade com os requisitos funcionais esperados.
	\item \textbf{Validação.} Os cenários de testes serão implementados na forma de provas de conceito para validação da proposta e do mecanismo implementado. O objetivo é conhecer a aderência do mecanismo em relação aos requisitos.
\end{alineas}