%!TEX root = Projeto.tex
\chapter{Motivação}
\section{Introdução}
A originação e o transporte de dados pela web são processos sujeitos a diversas preocupações relacionadas à segurança da informação. Seja na fronteira do usuário com a internet (o chamado \textit{front end}), na fronteira da internet com os sistemas subjacentes (\textit{back end}), bem como na infraestrutura entre as duas pontas, informações estão potencialmente expostas a adulteração e ao acesso não autorizado. Muitos dos riscos são mitigados com medidas aplicáveis durante o desenvolvimento, a publicação e a manutenção dos recursos que compõem as aplicações da web, tornando-a um território relativamente seguro e próspero para usuários e desenvolvedores das inúmeras aplicações bancárias, financeiras, comerciais, governamentais e sociais disponíveis através dos mais variados dispositivos. Ainda assim, uma parte da tecnologia que dá suporte a esses recursos é fundamentalmente limitada no nível de segurança da informação que pode oferecer, especialmente quando é considerado o rico ambiente de execução proporcionado pelos softwares navegadores.

Plataformas de \textit{front-end} como navegadores e ``apps'' expõem aos desenvolvedores uma variedade de tecnologias para a composição de \textit{experiências de usuário}. O \textit{front-end} é capaz de capturar informações  provenientes tanto do sistema local quanto da internet em diferentes domínios, combinar e exibir essas informações aos usuários, obter \textit{feedback} dos usuários e transmitir tudo isso para sistemas de \textit{back-end}. Essa flexibilidade permite a existência de aplicações \textit{mashup}, em que uma mesma interface é construída a partir de componentes publicados por diferentes provedores que são, a priori, independentes entre si e, por isso, não necessariamente pré-estabelecem qualquer relação de confiança com outros componentes que possam, em dado momento, coexistirem em uma página em um navegador.

Navegadores modernos oferecem recursos de expansão por softwares acessórios denominados ``plugins'' ou ``extensões''. Extensibilidade permite que o software acessório efetivamente tome controle de funcionalidades como gerenciamento de conexões de rede, eventos de navegação e o DOM. Softwares acessórios permitem que o usuário enriqueça sua experiência no navegador, incorporando recursos de produtividade, de multimídia e de integração com outros softwares. Ainda que dependam da permissão explícita do usuário no momento da incorporação ao navegador, os plugins e extensões podem se comportar como agentes de vazamento de informação de forma sutil e não necessariamente proposital.

Segurança da informação é um tópico importante na padronização das tecnologias que dão suporte à web. Protocolos de comunicação implementam criptografia, navegadores submetem seus módulos a restrições e limites de operação, e os desenvolvedores são desencorajados a expor informação sigilosa através das funcionalidades de programação. Mesmo assim, em seu todo a plataforma do navegador não oferece imunidade a falhas de segurança da informação, permitindo que dados do usuário sejam acessados e retransmitidos indevidamente.

A despeito de restrições impostas pelos navegadores para a minimização dos riscos à segurança da informação, o uso malicioso ou ingênuo de certas funcionalidades coloca em risco a integridade de aplicações \textit{mashup} e de extensões do navegador. A forma como os diversos componentes de \textit{front-end} são integrados abre vulnerabilidades:

\begin{alineas}
	\item \textit{scripts} de diferentes origens podem ser incorporados em uma mesma página, compartilhar informações e transmiti-las a outros domínios sem que o usuário esteja ciente;
	\item extensões de \textit{browser} podem ter acesso irrestrito ao DOM, monitorar as conexões à rede e modificar o funcionamento de funcionalidades sem o conhecimento do usuário;
	\item imagens, folhas de estilo e outros recursos podem ser incorporados como meio de efetuar ações e cometer vazamento de dados;
	\item trechos de \textit{scripts} podem ser incorporados sem o conhecimento do desenvolvedor e do usuário, permitindo a execução de código de vazamento de dados.
\end{alineas}

Sustentando essas vulnerabilidades está o fato de que, nas condições correntes, ser possível que um script efetue, indevidamente, acessos de leitura e escrita em informações que o usuário solicite (via requisições) ou informe (via formulários) para o navegador. Isso é derivado da forma inconsistente e parcial como a linguagem Javascript e as ``APIs'' do navegador tratam o isolamento entre scripts e dados de origens distintas. Este é um campo de pesquisas ativo, mas que ainda não resultou em práticas, ferramentas e protocolos de amplo alcance.




%Em suas mais de duas décadas de existência, a web e as tecnologias que lhe dão estrutura co-evoluíram com um variado conjunto de plataformas e linguagens. Essa evolução se manifesta não apenas na vasta gama de dispositivos de acesso e de infraestruturas de provimento de serviços, mas também na infinidade de aplicações cuja existência depende unicamente da web. No entanto, a despeito da evolução percebida em todos esses atores, alguns fundamentos da web mantêm relativa estabilidade: os protocolos de comunicação (HTTP e sua contrapartida segura HTTPS) e a arquitetura das plataformas-clientes (HTML, DOM, CSS, Javascript) são essencialmente os mesmos de 20 anos atrás. Essa afirmação não ignora as significativas evoluções que cada um dos componentes desse nível fundamental da web tem continuamente apresentado, mas apenas constata que o relacionamento entre eles mantém as mesmas premissas ao longo do tempo.

%Javascript é a linguagem utilizada para a codificação de programas -- denominados scripts -- que são executados pelos navegadores. Os scripts estão sujeitos a diversas restrições para que não comprometam a segurança e a funcionalidade do navegador: SOP, CORS e [etc...] são diretivas para neutralizar o efeito de scripts que poderiam, de outra forma, monitorar e utilizar indevidamente informações do usuário. Porém, essas diretivas são garantidas dentro de certos limites: em um nível extremo de restrições, não seria possível que os desenvolvedores criassem aplicações complexas como suítes de produtividade, redes sociais, serviços bancários e extensões do navegador. Especificamente, sites que empregam conteúdo e scripts provenientes de origens distintas não poderiam existir.

%Como outras linguagens, Javascript não oferece meios de se atribuir contextos de segurança e de confidencialidade às informações. Toda informação contida em objetos é visível por todos os scripts que compartilhem esse mesmo objeto. [Esta é uma preocupação que vem ganhando atenção nos últimos anos, sendo objeto de uma diversos estudos e iniciativas que pretendem minimizar ou extinguir os riscos à segurança da informação. -- ESTE TRECHO PARECE PARTE DA SOLUÇÃO, E NÃO DO PROBLEMA]

