 %!TEX root = Projeto.tex
\section{Motivação}

%ESTÁ MUITO CONFUSO. MELHOR COMEÇAR DIRETO AO PONTO COMO O DEIAN. DEIXAR EVIDENTE A DISTÂNCIA ENTRE O ESTADO DA ARTE E O ESTADO DAS FERRAMENTAS, E POR QUE ISSO COMPLICA A VIDA DE DESENVOLVEDORES E USUÁRIOS. ESCREVER PENSANDO QUE O OBJETIVO DO TRABALHO É ""AVALIAR"" UMA ALTERNATIVA AO ESTADO DA ARTE, SEM EFETIVAMENTE DIZÊ-LO AQUI.

É difícil desenvolver aplicações seguras para a web. Mesmo sob a defesa de \textit{firewalls}, protocolos de segurança, software antivírus e atualizações dos navegadores, as informações dos usuários permanecem fundamentalmente expostas ao acesso indevido por \textit{scripts} mal-intencionados ou mal-escritos. Essa condição se expressa de forma crítica por meio do conteúdo ativo, recurso do navegador que viabiliza a existência de aplicações interativas nas linguagens HTML e Javascript. Criar um \textit{script} com a intenção de ler o conteúdo potencialmente sigiloso de uma página da web e revelá-lo a terceiros não-autorizados é uma tarefa que exige pouca habilidade e que pode passar despercebida pelo aparato de segurança disponível.

Exemplos de manipulação maliciosa de scripts incluem sequestro de CDNs (CDN -- \textit{content distribution network}) \cite{Dorfman2013}, injeção de scripts por extensões, vírus ou redirecionamento \cite{Kinlan2015}, e ainda a inclusão de código não verificado pelo publicador \cite{Vanunu2016}. Tais tipos de ataques não tiram proveito de defeitos dos navegadores; ao invés disso, funcionam de acordo com os mecanismos de segurança da informação padronizados pela indústria. Isto ocorre porque parte da tecnologia que dá suporte ao conteúdo ativo é fundamentalmente limitada no nível de segurança da informação que pode oferecer, especialmente quando é considerado o rico ambiente de execução proporcionado pelo software navegador. Nele, a confidencialidade da informação é vulnerável a agentes maliciosos que operam em dois âmbitos:

\begin{alineas}
	\item Componentes de \textit{script} incorporados às páginas da web são executados com os mesmos privilégios e mesmo nível de acesso à página como um todo \cite[p. 2-3]{DeRyck2012}, sejam eles benignos ou não. O navegador oferece diretivas de segurança limitadas para conter o vazamento de informação entre componentes, uma vez que não há um meio de especificar relações de confiança entre eles \cite{Jang2010}.
	\item \textit{Plugins} e extensões do navegador têm nível de acesso mais elevado aos recursos de sistema do que aquele definido para componentes de páginas da web, permitindo que \textit{plugins} tomem controle de funcionalidades como gerenciamento de conexões de rede, eventos de navegação e o DOM. Ainda que dependam da permissão explícita do usuário no momento de sua instalação, os \textit{plugins} e extensões podem se comportar como agentes de vazamento de informação de forma sutil e não necessariamente proposital \cite{Heule2015}.
\end{alineas}

Na raiz das vulnerabilidades está a forma inconsistente e parcial com que a linguagem Javascript e as APIs do navegador tratam o isolamento entre scripts e dados. A caracterização e a solução desse problema fazem parte de um campo de pesquisas ativo \cite{Stefan2014}, \cite{Hedin2014}, \cite{Bichhawat2014}, \cite{Magazinius2014} e em busca por padronização \cite{W3C:WebAppSec}, mas que ainda não resultou em práticas, ferramentas e protocolos de amplo alcance pois dependem da adoção de políticas de segurança experimentais \cite{Hedin2014}, \cite{Bichhawat2014} e potencialmente degradantes de desempenho \cite[p. 14]{Stefan2014} por parte dos desenvolvedores de navegadores.

Os desenvolvedores de aplicações em conteúdo ativo, por sua vez, são limitados em suas opções para a publicação de \textit{sites} e páginas web seguras. Ainda que sigam as práticas recomendadas para a neutralização dos ataques mais comuns à segurança da informação (\cite{W3C:CORS}, \cite{W3C:SOP}, \cite{W3C:CSP}), o conteúdo dos componentes das páginas permanece acessível aos scripts incorporados e às extensões do navegador.

Buscando uma alternativa isenta de implementações experimentais, este trabalho investiga um método para o confinamento de informação através de recursos padronizados de HTML e Javascript. Com isso, seria possível encapsular informações em componentes interativos cuja estrutura é invisível tanto para o restante da página HTML como para ``plugins''. A conclusão deste trabalho se dará pela avaliação da efetividade do método como meio de ofuscamento de (1) informações incorporadas e exibidas na página e (2) do comportamento do usuário ao interagir com componentes HTML e Javascript desenvolvidos conforme o método proposto.

%A originação e o transporte de dados pela web são processos sujeitos a diversas preocupações relacionadas à segurança da informação. Seja na fronteira do usuário com a internet (o chamado \textit{front-end}), na fronteira da internet com os sistemas subjacentes (\textit{back-end}), bem como na infraestrutura entre as duas pontas, informações estão potencialmente expostas a adulteração e ao acesso não autorizado. Muitos dos riscos são mitigados com medidas aplicáveis durante o desenvolvimento, a publicação e a manutenção dos recursos que compõem as aplicações da web, tornando-a um território relativamente seguro e próspero para usuários e desenvolvedores. Ainda assim, uma parte da tecnologia que dá suporte a esses recursos é fundamentalmente limitada no nível de segurança da informação que pode oferecer, especialmente quando é considerado o rico ambiente de execução proporcionado pelo software navegador. Nele, a confidencialidade da informação está sujeita a ataques -- vazamentos -- provocados por agentes incorporados tanto nas páginas da web e quanto no próprio navegador.

%Agentes do primeiro caso tiram proveito de funcionalidades de HTML e Javascript para a composição de experiências de usuário em conteúdo misto, em que uma mesma interface (página da web) hospeda componentes publicados por diferentes provedores de serviços sem que estes tenham pré-estabelecido qualquer relação de confiança uns com os outros. Dado que esses componentes têm os mesmos privilégios e mesmo nível de acesso à página como um todo \cite[p. 2-3]{DeRyck2012}, caberia ao navegador aplicar diretivas de segurança que impedissem vazamentos de informação entre os componentes. Algumas dessas diretivas existem, mas sua aplicação é limitada \cite{Jang2010}.

%O segundo caso se baseia na extensibilidade do navegador. Navegadores modernos oferecem recursos de expansão por software acessório denominado \textit{plugins} ou extensões. O software acessório tem nível de acesso mais elevado aos recursos do navegador do que aquele definido para componentes de páginas da web, permitindo que \textit{plugins} tomem controle de funcionalidades como gerenciamento de conexões de rede, eventos de navegação e o DOM. Ainda que dependam da permissão explícita do usuário no momento da incorporação ao navegador, os \textit{plugins} e extensões podem se comportar como agentes de vazamento de informação de forma sutil e não necessariamente proposital \cite{Heule2015}.

%(REFORMULAR DAQUI EM DIANTE)

%Segurança da informação é um tópico importante na padronização das tecnologias que dão suporte à web. Protocolos de comunicação implementam criptografia, navegadores submetem seus módulos a restrições e limites de operação, e os desenvolvedores são desencorajados a expor informação sigilosa através das funcionalidades de programação. Mas a despeito de restrições impostas pelos navegadores para a minimização dos riscos à segurança da informação, o uso malicioso ou ingênuo de certas funcionalidades coloca em risco a integridade de aplicações \textit{mashup} e de extensões do navegador. A forma como os diversos componentes de \textit{front-end} são integrados abre vulnerabilidades:

%\begin{alineas}
%	\item \textit{scripts} de diferentes origens podem ser incorporados em uma mesma página, compartilhar informações e transmiti-las a outros domínios sem que o usuário esteja ciente;
%	\item extensões de navegador podem ter acesso irrestrito ao DOM, monitorar as conexões à rede e modificar o funcionamento de funcionalidades sem o conhecimento do usuário;
%	\item imagens, folhas de estilo e outros recursos podem ser incorporados como meio de efetuar ações e cometer vazamento de dados;
%	\item trechos de \textit{scripts} podem ser incorporados sem o conhecimento do desenvolvedor e do usuário, permitindo a execução de código projetado para vazamento de dados.
%\end{alineas}


