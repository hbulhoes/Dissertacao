%!TEX root = Projeto.tex
\section{Motivação}

% ---- NOVA RE-ESCRITA. INÍCIO
%Assim, tem que começar...a segurança da informação na web é há muito tempo uma preocupação..bla bla bla bla bla..conte histórias...cite vários trabalhos já realizados contra cross site scripting por exemplo. Depois... tem que mostrar que pesquisou. Isso mexerá nas referência também, mas tem que mexer. É um ponto crítico.

O software navegador é um ponto de convergência da informação na web. Seus usuários dependem dele para trabalhar, estudar e facilitar tarefas, e também para a comunicação e o entretenimento. Essas aplicações são causa e consequência da evolução do \poe{web browser}, que ao longo do tempo se transformou em uma plataforma para a distribuição de um tipo de software conhecido como ``aplicação web''. Hoje, muito distante de um simples navegador de conteúdo, o \poe{browser} provê recursos capazes não apenas de capturar, transmitir e apresentar informação, mas também de executar programas denominados de \scripts{}, destinados ao enriquecimento da experiência do usuário e ao compartilhamento de informação.

O amplo escopo das funcionalidades do navegador, potencializado por sua utilização maciça, torna as aplicações web alvos para ataques à confidencialidade e à integridade da informação que circula pela rede. A Fundação OWASP, organização internacional de apoio ao desenvolvimento de aplicações confiáveis, incluiu a exposição de dados confidenciais e a ocorrência de {\scripts} ``cross-site'' \poe{(XSS -- Cross-Site Scripring)} em sua lista das dez maiores ameaças à segurança das aplicações, edição 2017 \cite{OWASP:Top10}. Ambos os riscos se apoiam em vulnerabilidades do navegador para se manifestarem nas aplicações web. Em resposta, a comunidade de desenvolvimento dos \poe{browsers} incorporou ao software navegador diversos mecanismos para a defesa de seus usuários. Porém, problemas fundamentais, focos das vulnerabilidades, permanecem inerentes aos padrões de funcionamento que os navegadores precisam implementar, desta forma perpetuando riscos.

Com efeito, o software navegador é vetor de uma variedade de ataques à segurança da informação. Tais ataques são fundamentados em tecnologias que, ao mesmo tempo em que dão poder aos desenvolvedores de aplicações web, também tornam possíveis o vazamento e adulteração de dados. O navegador é flexível a ponto de possibilitar que uma mesma página comporte elementos de composição publicados por desenvolvedores distintos. Isso permite que recursos construídos com Javascript, \poe{frames}, aplicativos Java e conteúdo em Flash coexistam e levem a experiência de usuário na web a um patamar mais elevado em termos de funcionalidade, porém mais perigoso no que diz respeito ao conjunto de vulnerabilidades aproveitáveis por fraudadores e \poe{hackers}. Ademais, esses elementos de página também não foram concebidos com políticas de segurança claras e consistentes. Nessas circunstâncias, a segurança das aplicações web complexas precisa se basear na confiança que seus autores depositam nos recursos de terceiros incorporados às aplicações.

Aplicativos para Java e conteúdo em Flash, reiteradamente vulneráveis e em descompasso com a evolução da web, acabaram em desuso e deixaram de ter suporte nativo nos navegadores \cite{Verge2016, Adobe2017}. Javascript, ao contrário, cresceu em funcionalidade e popularidade entre os desenvolvedores, impulsionando uma forma de aplicação web marcada pelo ``conteúdo ativo''. O conteúdo ativo tira proveito de uma característica dos navegadores chamada de DOM (\poe{Document Object Model}), uma interface de programação que permite a manipulação do conteúdo das páginas por {\scripts}. O DOM, presente nos navegadores desde a introdução da linguagem Javascript e continuamente enriquecido com funcionalidades, é a base em que se apoia a construção de qualquer aplicação web moderna.

É relevante reiterar que {\scripts} podem ser carregados pelo navegador a partir de múltiplas origens -- \poe{sites} diferentes daquele em que foi publicada a página que fizer uso deles. Um exemplo desse tipo de conteúdo pode ser encontrado em \poe{blogs} e \poe{sites} jornalísticos, dentro dos quais podem ser incorporados anúncios provenientes de \poe{sites} de serviços publicitários. Nesses casos, o navegador cria um contexto de execução compartilhado onde todos os {\scripts}, independentemente de suas origens, têm acesso a toda informação contida nas páginas e, caso tenham sido desenvolvido sem cuidado ou com finalidades duvidosas, podem interferir no funcionamento uns dos outros. Desta característica podem nascer alguns dos problemas que motivam a existência deste trabalho.

Um dos problemas, o já citado \poe{cross-site scripting} (XSS), caracteriza-se pelo redirecionamento de informação sigilosa para \poe{sites} não confiáveis. Serviços como MySpace.com e Twitter já foram alvo desse tipo de ataque \cite{IBM2017}, assim como o comércio eletrônico E-Bay \cite{Vanunu2016}. Contra XSS não existe uma forma universal de prevenção: cada aplicação web deve se precaver para que {\scripts} maliciosos não sejam ativados. Mas ainda que as devidas precauções sejam tomadas, nada impedirá que o navegador carregue {\scripts} adulterados sem conhecimento dos autores de uma página, como ocorreu com o \poe{bureau} de crédito norte-americano Equifax \cite{Segura2017} e na invasão e adulteração de {\scripts} da rede de distribuição de conteúdo BootstrapCDN \cite{Dorfman2013}. Em ambos os casos, os servidores de hospedagem de alguns {\scripts} foram invadidos e passaram a publicar código Javascript impróprio, diferente daquele originalmente tido como confiável. %Mesmo as extensões do navegador proporcionam outros meios de ataques, rotineiramente explorados \cite{Forrest2017}.

% Contextualizar: por que é importante um esforço para ajudar o desenvolvedor na proteção do conteúdo no navegador?
Assim, o desenvolvimento de uma aplicação segura para a web demanda esforços para que seja evitada a exposição e a manipulação indevidas das informações do usuário. Para esse propósito, o desenvolvedor conta com um conjunto de práticas e recomendações padronizadas, efetivamente protegendo a aplicação e seus usuários de uma série de vulnerabilidades. Pelo lado dos desenvolvedores de navegadores, o estabelecimento de conjuntos de regras como a SOP (\poe{same-origin policy}), e de protocolos como o HTTPS e CORS (\poe{cross-origin resource sharing}) elevam a capacidade do navegador em manter um ambiente de execução seguro. Essas tecnologias são produto de iniciativas da comunidade de desenvolvimento dos navegadores, das aplicações web e das organizações padronizadoras da inernet, como os comitês W3C e IETF. %Ambos se baseiam na noção do ``domínio'' como identificador da origem e, por consequência, da confiabilidade de um recurso: o domínio da página, denotado pela combinação do \poe{protocolo}, \poe{nome do host} e \poe{porta TCP} de onde o navegador requisitou o conteúdo carregado, é tido como o mais confiável, enquanto recursos requisitados de domínios diferentes são considerados menos confiáveis.
%CORS, por exemplo, foi destacado por \cite{DeRyck2012} em sua avaliação de mecanismos de segurança no navegador, na qual diversos dos mecanismos disponíveis são comparados segundo quatro categorias de requisitos -- Separação, Interação, Comunicação e Controle do Comportamento.

Contudo, a manutenção de um ambiente de execução seguro em uma página web ainda se sustenta dentro da condição de que todo conteúdo ativo carregado por uma aplicação esteja sob o conhecimento e confiança de seu desenvolvedor. Centralmente, o DOM, estrutura volátil em que persistem informações dos usuários, permanece exposto a {\scripts} mal-intencionados ou mal-escritos, executados em contexto da página ou como extensões do navegador \cite{Heule2015_Most_Dangerous_Code}. Apesar do aparato de segurança presente no navegador, criar um {\script} destinado a ler o conteúdo potencialmente sigiloso no DOM e revelá-lo a terceiros não autorizados é uma tarefa que exige pouca habilidade e que pode passar despercebida.

% Contextualizar: o que tem sido feito nessa direção, em termos de experimentos?
Uma das formas propostas para a solução dessas inconsistências seria a introdução de um modelo de segurança que permitisse o monitoramento do fluxo da informação na linguagem Javascript \cite[p.3]{Heule2015_IFC_Inside}. Dessa forma, toda manipulação de dados dependeria de uma validação dos contextos de segurança atribuídos ao dado e aos participantes envolvidos. Isso impediria, por exemplo, que informações sensíveis armazenadas em um nó do DOM fossem lidas e transmitidas para endereços da web desautorizados -- um impedimento que seria obedecido não de forma discricionária, imperativa, mas de modo mandatório, declarativo e integrado ao \poe{runtime} da linguagem. Essa abordagem, conhecida pela sigla IFC (\poe{information flow control}), não é compatível com a implementação da linguagem Javascript incluída nos navegadores existentes. Incipientemente, mecanismos de suporte ao IFC são implementados em navegadores experimentais \cite{Hedin2014, Bichhawat2014}, o que impede sua adoção maciça.

% Contextualizar: como esta proposta contribui?
Assim, se os meios existentes para proteção contra o vazamento de informação contida no DOM são insuficientes, e se um mecanismo robusto como o IFC ainda não tenha sido incorporado aos navegadores tradicionais, pode ser importante propor uma abordagem que desse ao desenvolvedor um mecanismo, ainda que discricionário, o qual implementasse uma barreira que impedisse o acesso não prescrito aos nós, ou regiões, do DOM, a critério do desenvolvedor. Isso tornou-se uma possibilidade com a introdução de APIs como a de suporte ao \poe{Shadow DOM} \cite{W3C:ShadowDOM}, que fornecem um grau de inviolabilidade de acesso à informação em determinados contextos de programação de \scripts{} no navegador. Aplicando esses recursos, este trabalho propõe uma melhoria da segurança da informação baseada em APIs padronizadas, imediatamente disponíveis aos desenvolvedores e usuários, sem lançar mão de tecnologias experimentais.