 %!TEX root = Projeto.tex
\section{Motivação}

% ---- NOVA RE-ESCRITA. INÍCIO
%Assim, tem que começar...a segurança da informação na web é há muito tempo uma preocupação..bla bla bla bla bla..conte histórias...cite vários trabalhos já realizados contra cross site scripting por exemplo. Depois... tem que mostrar que pesquisou. Isso mexerá nas referência também, mas tem que mexer. É um ponto crítico.

Segurança da informação na internet é um tópico de preocupação desde antes do início da exploração comercial da rede. Ainda nos anos 1970, já circulavam ideias a respeito de como tornar os protocolos de comunicação mais seguros, e de como fazer com que os acessos pudessem ser mais restritos \cite{WSP2015}. Porém, foi depois da popularização da web nos anos 1990 que a segurança da informação deixou de ser um assunto de especialistas e entrou no dia-a-dia do público leigo, que passou a conviver com ameaças de vírus distribuídos por e-mail, fraudes bancárias \textit{online}, interceptação das comunicações e \textit{backdoors} em aplicativos tidos como confiáveis, entre outras preocupações.

Podemos destacar o software navegador, ou \textit{browser} da web, como vetor de uma variedade de ataques à segurança da informação. No início, a grande ameaça ao vazamento de dados era o próprio meio por onde a informação trafegava, já que o protocolo HTTP não oferece dificuldades para ser decodificado por qualquer ente mal-intencionado entre dois \textit{hosts} na web. HTTP com SSL solucionou esse problema, mas a essa altura o navegador já começava dar suporte a algumas tecnologias que iriam tornar possíveis novas formas de vazamento e adulteração de dados.

Elementos de composição de página como Javascript, \textit{frames}, aplicativos Java e conteúdo em Flash entraram em cena quase simultaneamente, levando a experiência de usuário na web para um patamar mais elevado em termos de funcionalidade, porém mais perigoso no que diz respeito ao conjunto de vulnerabilidades aproveitáveis por fraudadores e \textit{hackers}. Esses elementos de página, além dos \textit{bugs} ocasionais que se manifestavam de diferentes maneiras, também não foram concebidos com políticas de segurança claras e consistentes.

Aplicativos para Java e Flash acabaram em desuso, e os navegadores deixaram de oferecer suporte nativo a essas tecnologias \cite{Verge2016, Adobe2017}. Javascript, ao contrário, cresceu em funcionalidade e popularidade entre os desenvolvedores, impulsionando uma forma de aplicação denominado \textit{conteúdo ativo}. O conteúdo ativo tira proveito de uma característica dos navegadores chamada de DOM (\textit{Document Object Model}), uma interface de programação que permite a manipulação do conteúdo das páginas por \scripts, os quais podem ser incorporados pelo navegador a partir de \textit{sites} diferentes. Um exemplo desse tipo de conteúdo pode ser encontrado em \textit{sites} jornalísticos, dentro dos quais podem ser incorporados anúncios provenientes de \textit{sites} de serviços publicitários. Nesses casos, o navegador cria um contexto de execução onde \scripts de múltiplas origens compartilham das informações contidas nas páginas. Desta característica surgiram alguns dos problemas que motivam a existência deste trabalho.

Um dos problemas, conhecido como \textit{cross-site scripting} (XSS), caracteriza-se pelo redirecionamento de informação sigilosa para \textit{sites} não confiáveis. \textit{Sites} como MySpace.com e Twitter já foram alvo desse tipo de ataque \cite{IBM2017}, assim como o site E-Bay \cite{Vanunu2016}. Contra XSS não existe uma forma universal de prevenção -- cada aplicação web deve se precaver para evitar a ativação de \scripts maliciosamente incorporados ao seu conteúdo. Mas ainda que as devidas precauções sejam tomadas, o navegador está sujeito a carregar \scripts adulterados sem conhecimento dos autores de uma página, como ocorreu com o \textit{bureau} de crédito norte-americano Equifax \cite{Segura2017} e na invasão e adulteração de \scripts da rede de distribuição de conteúdo BootstrapCDN \cite{Dorfman2013}. Em ambos os casos, os servidores de hospedagem de alguns \scripts foram invadidos e passaram a publicar conteúdo ativo impróprio. Mesmo as extensões do navegador proporcionam outros meios de ataques, rotineiramente explorados \cite{Forrest2017}.

% Contextualizar: por que é importante um esforço para ajudar o desenvolvedor na proteção do conteúdo no navegador?
Assim, o desenvolvimento de uma aplicação segura para a web demanda esforços para que seja evitada a exposição e a manipulação indevidas das informações do usuário. Para esse propósito, o desenvolvedor conta com um conjunto de práticas e recomendações estabelecidas, efetivamente protegendo a aplicação e seus usuários de uma série de vulnerabilidades. Conjuntos de regras, como a SOP (\textit{same-origin policy}), e protocolos como o CORS (\textit{cross-origin resource sharing}) elevam a capacidade do navegador em manter um ambiente de execução seguro.%Ambos se baseiam na noção do ``domínio'' como identificador da origem e, por consequência, da confiabilidade de um recurso: o domínio da página, denotado pela combinação do \textit{protocolo}, \textit{nome do host} e \textit{porta TCP} de onde o navegador requisitou o conteúdo carregado, é tido como o mais confiável, enquanto recursos requisitados de domínios diferentes são considerados menos confiáveis.
%CORS, por exemplo, foi destacado por \cite{DeRyck2012} em sua avaliação de mecanismos de segurança no navegador, na qual diversos dos mecanismos disponíveis são comparados segundo quatro categorias de requisitos -- Separação, Interação, Comunicação e Controle do Comportamento.

Contudo, tal ambiente é protegido dentro da condição de que todo conteúdo ativo carregado por uma aplicação está sob o conhecimento e confiança de seu desenvolvedor. Dentro da estrutura de documento da página web as informações dos usuários permanecem fundamentalmente expostas a \scripts mal-intencionados ou mal-escritos, executados em contexto da página ou como extensões do navegador \cite{Heule2015_Most_Dangerous_Code}. Criar um \script destinado a ler o conteúdo potencialmente sigiloso de uma página da web e revelá-lo a terceiros não autorizados é uma tarefa que exige pouca habilidade e que pode passar despercebida pelo aparato de segurança disponível, incluindo as restrições de SOP e CORS.

% Contextualizar: o que tem sido feito nessa direção, em termos de experimentos?
Uma das formas propostas para a solução dessas inconsistências seria a introdução de um modelo de segurança que permitisse o monitoramento do fluxo da informação na linguagem Javascript \cite[p.3]{Heule2015_IFC_Inside}. Dessa forma, toda cópia, referência ou manipulação de dados na linguagem dependeria de uma validação dos contextos de segurança atribuídos ao dado e aos recipientes envolvidos. Isso impediria, por exemplo, que informações sensíveis presentes em um nó do DOM fossem lidas e transmitidas para endereços da web desautorizados -- um impedimento que não seria obedecido de forma discricionária, imperativa, mas de modo mandatório, declarativo e integrado ao \textit{runtime} da linguagem. Essa abordagem, conhecida pela sigla IFC (\textit{information flow control}), não é compatível com os navegadores existentes. Mecanismos de suporte ao IFC são implementados como navegadores experimentais \cite{Hedin2014, Bichhawat2014}, o que impede sua adoção maciça.

% Contextualizar: como esta proposta contribui?
Assim, se os meios existentes para proteção contra o vazamento de informação contida no DOM são insuficientes, e se um mecanismo robusto como o IFC ainda não foi incorporado aos navegadores tradicionais, pode ser importante propor uma abordagem que desse ao desenvolvedor um mecanismo que, embora discricionário, implementasse uma barreira que impedisse o acesso não prescrito aos nós, ou regiões, do DOM, a critério do desenvolvedor. Isso tornou-se uma possibilidade com a introdução de APIs como a de suporte ao \textit{Shadow DOM} \cite{W3C:ShadowDOM}, que fornecem um grau de invisibilidade a determinados conteúdos de HTML. Propor uma melhoria da segurança da informação baseada em APIs padronizadas, imediatamente disponíveis aos desenvolvedores e usuários, é a proposta deste trabalho.