%!TEX root = Projeto.tex
\section{Organização do trabalho}

%A seção 1, Introdução, fundamenta este trabalho pela caracterização do problema que motivou sua existência, pela definição de um objetivo relevante no domínio do problema, e pelas contribuições que o trabalho se propõe a fazer para o corpo de conhecimento do tema da segurança da informação no navegador com Javascript e HTML.

A seção 2, Estado da Arte, enquadra o tema sob três pontos de vista: (1) das vulnerabilidades derivadas da tecnologia atual, (2) dos recursos implementados pelos navegadores para a detenção de determinados ataques à segurança da informação, e (3) das propostas experimentais para a mitigação de vulnerabilidades. O panorama formado por esses três pontos de vista corresponde ao contexto em que as contribuições deste trabalho estão inseridas.

A seção 3, Proposta, descreve um método para o desenvolvimento de componentes de HTML que mantenham invisíveis, para o restante da página, as informações mantidas ou geradas por esses componentes, ao mesmo tempo em que expõe uma interface de programação baseada em controle do acesso à informação encapsulada. São apresentadas nesta seção a disponibilidade dos recursos necessários para a implementação do método, bem como suas limitações de uso.

Na seção 4, Avaliação, são propostos critérios para a verificação da eficácia do método proposto: disponibilidade nas plataformas de navegação, limites de proteção versus vulnerabilidades mitigadas, e requisitos de funcionamento. A seção se completa com a aplicação desses critérios sobre o método proposto, em comparação com trabalhos embasados pela abordagem de IFC -- o controle de fluxo de informação define, no âmbito do problema, maior granularidade na segurança da informação em Javascript, ao custo da compatibilidade com a base instalada de navegadores.

O conteúdo da seção 5, Conclusões, deriva da reflexão crítica sobre a implementação do método proposto em contraponto aos resultados observados na avaliação qualitativa. Recomendações sobre a aplicação do método, além de oportunidades a serem exploradas por trabalhos futuros, fecham a conclusão dos esforços deste trabalho.