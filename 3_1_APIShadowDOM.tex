% !TeX spellcheck = <none>
%!TEX root = Projeto.tex

\section{API de ocultação do DOM fundamentada em Shadow DOM}
O mecanismo de ocultação do DOM proposto neste trabalho faz com que a informação contida em um ramo específico da árvore do DOM, bem como a informação gerada por esse ramo do DOM, esteja fora do alcance de \scripts{} que se baseiem em interações com o DOM para obter dados que venham a ser indevidamente propagados. Trata-se de uma medida discricionária de segurança da informação, \dubious{caracterizável pelos aspectos X, Y e Z das políticas discricionárias de segurança da informação}.

A característica de ocultação deriva de uma propriedade das APIs de \textit{shadow DOM}, que estabelece, em sua especificação, que para todo \textit{shadow host} será especificada a visibilidade externa de sua \textit{shadow tree} (estrutura de DOM subjacente associada ao \textit{host}). Essa propriedade é determinada no momento da criação do \textit{shadow host}, conforme exemplificado na listagem de código \ref{Src: jsShadowHostCtor}, linhas 2 e 6.

\lstinputlisting[language=JavaScript,
inputencoding=utf8,
label={Src: jsShadowHostCtor},
caption={Determinação da visibilidade da \textit{shadow tree} e sua consequência na tentativa de acesso ao DOM subjacente}]{codigo/sample04-shadow-host-ctor.js}

Dessa forma, o método {\ttfamily{Element.attachShadow(\{mode: open\})}} abre a possibilidade de que uma \textit{shadow tree} seja inacessível a \scripts{} que não detenham uma referência à \textit{shadow root}, efetivamente provendo um meio programático para a definição de regiões protegidas na estrutura DOM da página web. No entanto, a invisibilidade do conteúdo da \textit{shadow root} não é de qualquer modo garantida pela API de Shadow DOM. A extensibilidade da linguagem Javascript permite que \scripts{} interceptem os métodos da API, ganhando acesso a todas as \textit{shadow roots} criadas em uma página web. Essa estratégia é detalhada na seção seguinte. %exemplificada na listagem de código \ref{Src: jsShadowApiIntercept}.

\subsection{Interceptação de métodos nativos do ambiente de execução}

A linguagem Javascript incorpora o conceito de herança prototípica para permitir que programas escritos nessa linguagem façam uso de funcionalidades da programação orientada a objetos como herança e polimorfismo. Protótipos de objeto são semelhantes à ideia de ``classes'' em linguagens estáticas com suporte à POO, porém, ao contrário delas, Javascript permite a redefinição dinâmica dos membros dos protótipos/classes. A listagem \ref{Src: jsPrototypicalInheritance} exemplifica esse mecanismo.

%Objetos em Javascript são entidades de estrutura dinâmica, comportando-se como dicionários ou \textit{hashes} de informação. Por meio de herança prototípica, objetos em Javascript são definidos como uma composição formada tanto pelos atributos herdados da cadeia de protótipos quanto dos atributos associados diretamente ao objeto.

\lstinputlisting[language=JavaScript,
inputencoding=utf8,
label={Src: jsPrototypicalInheritance},
caption={Demonstração de herança prototípica em Javascript}]{codigo/sample05-prototype.js}

Uma consequência da flexibilidade oferecida pela herança prototípica é que um \script{} pode modificar os métodos de um protótipo de forma que o acionamento desses métodos seja encapsulado por código arbitrário. Essa capacidade de interceptação permite que métodos sejam instrumentados, monitorados ou corrompidos para quaisquer fins. O exemplo na listagem \ref{Src: jsMethodHijacking} demonstra como o método {\texttt{Number.toString()} pode ser desvirtuado para retornar informação incorreta -- neste caso, o método sempre retornará a mesma \textit{string} ``42''.

\lstinputlisting[language=JavaScript,
inputencoding=utf8,
label={Src: jsMethodHijacking},
caption={Exemplo de método nativo interceptado por um script}]{codigo/sample06-hijack.js}