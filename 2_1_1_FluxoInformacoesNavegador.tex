%!TEX root = Projeto.tex
\subsection{Fluxo da informação no navegador}
\begin{todo}
Pontos a considerar:

\begin{enumerate}
	\item O navegador da web é um cliente da internet.
	\item O ciclo de vida de uma página é "curto" (cada página, uma aplicação diferente? um contexto de execução diferente?)
	\item Onde está a informação no navegador, e qual sua duração possível?
	\begin{enumerate}
		\item Endereço (URL) corrente
		\item Requisição (pode ser da página como de "sub-recursos" como imagens, iframes, scripts, estilos, xhr...)
		\item Resposta
		\item Estrutura da página (DOM)
		\item Eventos do navegador/DOM
		\item Closures
		\item Pilha de chamadas
		\item Extensões do navegador
		\item Caches
		\item Cookies
		\item Local DB
		\item Em HTTPS, temos mais fluxos?
	\end{enumerate}
\end{enumerate}

Então será possível indicar quais dos fluxos estão protegidos contra vazamento de informação, e como estão. E consequentemente quais fluxos não estão cobertos.
\end{todo}
