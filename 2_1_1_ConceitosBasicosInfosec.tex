%!TEX root = Projeto.tex
\section{Conceitos básicos}

\subsection{Segurança da informação}
Segundo \cite{ISO2016}, segurança da informação é um processo com os objetivos de ``preservação da confidencialidade, integridade e disponibilidade da informação''. \cite{Foster1998} elabora esses objetivos, descrevendo a confidencialidade como a condição na qual a informação só pode ser acessada pelos agentes autorizados, integridade como a capacidade de proteger a informação contra modificações não autorizadas, e a disponibilidade como a capacidade de garantir acesso à informação quando necessário; \cite{Foster1998} ainda atribui mais duas características a um sistema de segurança da informação: \textit{accountability} como a possibilidade de se atribuir um agente para cada ação ocorrida dentro do sistema, e \textit{assurance} como o grau de confiabilidade na segurança do sistema em relação aos seus objetivos declarados.

Neste trabalho, qualquer definição de segurança da informação será restrita aos sistemas de informação relacionados com a navegação de usuários através da web: provedores de serviço (\textit{sites}, servidores da web), protocolos de comunicação em rede (HTTP, HTTPS, \textit{web sockets}), navegadores (\textit{browsers}) e os ambientes de execução de Javascript embutidos nos navegadores. Isto delimita a área de conhecimento relevante para este trabalho.

%\subsection{Políticas de controle de acesso}
%Segundo \cite{Goguen1982}, uma política de controle de acesso é necessária para que se estabeleçam quais fluxos de dados serão permitidos em um sistema de informação.

\subsection{Modelos de controle de acesso}
Enquanto a definição dos requisitos de segurança da informação estabelece seus objetivos, os modelos definem os sistemas derivados desses objetivos \cite{Goguen1982}. \cite{Foster1998} menciona diferentes modelos de controle de acesso, categorizados de modo amplo como modelos discricionários (DAC -- \textit{discretionary access control}) e mandatórios (MAC -- \textit{mandatory access control}). Modelos discricionários se baseiam na definição dos relacionamentos de segurança entre agentes e objetos em um sistema, como, por exemplo, a política de que um {\script} -- a parte \textit{agente} -- não pode iniciar conexões com domínios diferentes do seu próprio -- a parte \textit{objeto}. Modelos discricionários são os mais comumente utilizados para estabelecer mecanismos de segurança nos navegadores. O campo de atuação desses modelos é limitado aos relacionamentos de segurança estabelecidos, e portanto não podem garantir a segurança da informação quando esta ultrapassa o domínio desses relacionamentos. Isto significa, por exemplo, que dados legitimamente obtidos dentro de regras discricionárias pode ser replicado para um contexto não-seguro sem qualquer impedimento derivado do modelo de segurança.

Modelos mandatórios não atribuem explicitamente as regras de controle de acesso aos objetos e agentes de um sistema. Ao invés disso, estabelecem níveis de confidencialidade utilizados para classificar os participantes do sistema de informação, viabilizando o controle dinâmico do trânsito da informação entre os agentes. Num modelo mandatório, o nível de segurança de um dado impede que ele seja obtido ou modificado por agentes com níveis de segurança mais baixos. O controle do fluxo da informação faz dos MACs modelos mais robustos do que os DACs \cite{Foster1998}.

\subsection{Controle do fluxo de informações}
O controle do fluxo de informações (IFC -- \textit{information flow control}) é um mecanismo que atua, em tempo de execução, nos meios de propagação dos valores entre os espaços de armazenamento de um sistema computacional de modo a impedir fluxos não autorizados dos dados \cite{Denning1976}. IFC é um modelo de controle de acesso do tipo mandatório e baseia-se em \textit{classes de segurança} ``altas'' e ``baixas'', simbolizadas pelas letras \texttt{<h>} e \texttt{<l>}, respectivamente, para indicar graus de confidencialidade das informações e dos seus espaços de armazenamento (\textit{heap}, pilha, redes, dispositivos etc). Operações entre entidades com classes de segurança diferentes, como a cópia do valor de uma variável \texttt{<h>} (confidencial) para a variável \texttt{<l>} (pública), são automaticamente impedidas de prosseguir.

IFC distingue entre fluxos de informação explícitos e implícitos. Um fluxo explícito ocorre quando uma informação classificada como ``alta'' é diretamente copiada para um contexto de classificação ``baixa'', como na listagem de código \ref{Src: jsIFCExplicitFlow}. Em um fluxo implícito, não é a informação em si que transita entre contextos de classificação diferente, mas sim alguma informação derivada dela através da qual seja possível fazer qualquer inferência sobre seu conteúdo. Um exemplo de fluxo implícito encontra-se la listagem \ref{Src: jsIFCImplicitFlow}. Um mecanismo que suporte IFC deve ser capaz de interromper vazamento de informação em ambos os tipos de fluxo.

\lstinputlisting[language=JavaScript,
inputencoding=utf8,
label={Src: jsIFCExplicitFlow},
caption={Vazamento de dados em fluxo explícito de informação}]{codigo/sample02-ifc-implicit.js}

\lstinputlisting[language=JavaScript,
inputencoding=utf8,
label={Src: jsIFCImplicitFlow},
caption={Vazamento de dados em fluxo implícito de informação}]{codigo/sample03-ifc-explicit.js}