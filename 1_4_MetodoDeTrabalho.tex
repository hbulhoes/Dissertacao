%!TEX root = Projeto.tex
\section{Método de trabalho}

%COMO ESTE TRABALHO VAI ALCANÇAR O OBJETIVO E CONCRETIZAR AS CONTRIBUIÇÕES?

O método para encapsulamento de informação por \textit{shadow DOM} resultante deste trabalho será o produto de uma sequência de tarefas para a exploração do problema e elaboração da solução. O método de trabalho, então, é composto das atividades enumeradas a seguir.


\begin{alineas}
	\item \textbf{Coleta de evidências}
	Nesta primeira atividade, os problemas abordados pelo método proposto serão materializados em simulações e casos de teste que evidenciem acessos não autorizados a informações contidas em componentes de páginas da web. Especificamente, as evidências devem provar que é possível efetuar as seguintes ações sem o conhecimento ou consentimento do usuário:
	
	\begin{alineas}
		\item scripts provenientes de domínios diferentes podem observar o conteúdo de páginas da web, incluindo identificações, senhas, códigos de cartão de crédito e números de telefone, desde que essas informações estejam presentes na estrutura da página;
		\item scripts agindo em extensões do navegador podem observar o conteúdo de páginas da web e de seus \textit{iframes};
		\item scripts de qualquer natureza podem registrar o comportamento do usuário ao interagir com a página, capturando eventos de teclado e de mouse;
		\item scripts de qualquer natureza conseguem interceptar funcionalidades do navegador para extrair informações que transitem pelas interfaces de programação do DOM e de Javascript.
	\end{alineas}
	
	Artefatos derivados dos casos de testes, codificados como páginas da web, serão insumo das atividades de avaliação do método proposto. Parte desta tarefa será dedicada à automação dos casos de teste pelo \textit{framework} de programação de testes Selenium\footnote{http://www.seleniumhq.org/}. Os scripts de automação serão agregados ao conjunto de artefatos de código derivados deste trabalho.
	
	\item \textbf{Proposição do método}
	O objetivo desta atividade é projetar um método de encapsulamento com o qual desenvolvedores de páginas da web possam definir componentes de HTML imunes ao vazamento de informação por meio de Javascript. O método precisará atender aos seguintes requisitos:
	
	\begin{alineas}
		\item Permitir que qualquer combinação de elementos HTML seja encapsulável em um componente inviolável por scripts externos a si;
		\item Ser compatível com bibliotecas e \textit{frameworks} de desenvolvimento em Javascript, HTML e CSS;
		\item Estabelecer um protocolo de confiança entre si e agentes (scripts) externos;
		\item Sob esse mesmo protocolo, expor uma interface de programação para a leitura e modificação das informações encapsuladas;
		\item Ser compatível com recursos padronizados e não-experimentais de HTML e Javascript.
	\end{alineas}
	
	O resultado desta atividade é uma especificação técnica da solução proposta, incluindo requisitos e limitações de uso.
	
	\item \textbf{Implementação do método}
	Esta atividade tem a finalidade de produzir um componente de HTML e Javascript compatível com os requisitos estabelecidos pela proposta, seguido pela incorporação desse artefato às páginas web derivadas dos casos de teste.
	
	\item\textbf{Avaliação do método}
	Nesta tarefa, o componente implementado será submetido à avaliação de sua eficácia frente às vulnerabilidades e sua compatibilidade em relação aos requisitos do método. Esse resultado será verificado pelo acionamento dos testes automatizados, que demonstrarão se os problemas evidenciados são neutralizados pela implementação.
	
	\item \textbf{Síntese dos resultados}
	O produto do acionamento dos testes indicará não apenas o sucesso ou falha em cada caso de teste, mas também fornecerá informações como a diferença no desempenho (medido em função do tempo de execução e memória do navegador) e nos erros ou exceções capturadas em tempo de execução, antes e depois da aplicação do método. Eventualmente, falhas de segurança que não forem mitigadas serão diagnosticadas com base nas especificações das APIs utilizadas nos testes e na implementação -- isto é, será avaliado se a falha de segurança é uma manifestação do comportamento esperado daquela API. A partir dessas observações será elaborada uma síntese dos resultados alcançados, incluindo uma reflexão sobre a abrangência e utilidade dos artefatos produzidos. As realizações e limitações deste trabalho serão comparadas com aquelas encontradas em trabalhos relacionados experimentais como \cite{Hedin2014} e \cite{Stefan2014}, posicionando definitivamente a contribuição deste trabalho dentro do panorama de segurança da informação.
\end{alineas}
