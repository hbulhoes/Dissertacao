%!TEX root = Projeto.tex
\section{Método de trabalho}

%COMO ESTE TRABALHO VAI ALCANÇAR O OBJETIVO E CONCRETIZAR AS CONTRIBUIÇÕES?

Os objetivos deste trabalho serão o produto de uma sequência de atividades dedicadas à exploração do problema e elaboração da solução. O método de trabalho, então, é composto das atividades enumeradas a seguir.


\begin{alineas}
	\item \textbf{Levantamento bibliográfico.}
	Esta atividade realiza-se pela pesquisa de trabalhos relacionados à segurança da informação no software navegador, incluindo contribuições relevantes que deram embasamento a esses trabalhos.
	
	\item \textbf{Levantamento de situações de risco envolvendo vazamento de informação.}
	Nesta atividade, os problemas-alvo motivadores deste trabalho serão materializados em simulações e casos de teste. As evidências deverão provar que é possível efetuar as seguintes ações sem o conhecimento ou consentimento do usuário:
	
	\begin{alineas}
		\item \scripts provenientes de domínios diferentes podem observar o conteúdo de páginas da web, incluindo identificações, senhas, códigos de cartão de crédito e números de telefone, desde que essas informações estejam presentes no DOM;
		\item \scripts agindo em extensões do navegador podem observar o conteúdo de páginas da web e de seus \textit{iframes};
		\item \scripts de qualquer natureza podem registrar o comportamento do usuário ao interagir com a página, capturando eventos de teclado e de mouse;
		\item \scripts de qualquer natureza conseguem interceptar APIs e com isso extrair informações que transitem pelas interfaces de programação do DOM e da linguagem Javascript.
	\end{alineas}

	\item \textbf{Análise de requisitos.}
	O levantamento bibliográfico efetuado fornecerá o insumo para a enumeração dos requisitos funcionais e não-funcionais relevantes para as situações de risco levantadas.
	
	\item \textbf{Proposição.}
	O objetivo desta atividade é elaborar um método para que os objetivos do trabalho sejam alcançados, em aderência aos requisitos funcionais e não-funcionais estabelecidos.

	\item \textbf{Implementação.}
	Esta atividade tem a finalidade de produzir um componente de HTML e Javascript compatível com os requisitos estabelecidos pela proposta.
		
	\item\textbf{Avaliação do método.}
	Nesta tarefa, o componente implementado será submetido à avaliação de sua eficácia frente às vulnerabilidades, e de sua compatibilidade em relação aos requisitos do método.
	
	\item \textbf{Síntese dos resultados.}
	A partir das observações produzidas na atividade de avaliação, será elaborada uma síntese dos resultados alcançados em contraste com os objetivos desta proposta.
\end{alineas}
