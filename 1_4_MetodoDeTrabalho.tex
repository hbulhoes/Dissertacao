%!TEX root = Projeto.tex
\section{Método de trabalho}

%COMO ESTE TRABALHO VAI ALCANÇAR O OBJETIVO E CONCRETIZAR AS CONTRIBUIÇÕES?

Para validar a proposta de Encapsulamento por \textit{shadow DOM}, propõe-se uma sequência de tarefas para a exploração do problema e elaboração da solução. O método de trabalho, então, é composto das atividades enumeradas a seguir.


\subsection{Coleta de evidências}
Nesta primeira atividade, os problemas endereçáveis pelo método proposto serão materializados em simulações e casos de teste que evidenciem acessos não autorizados a informações contidas em componentes de páginas da web. Especificamente, as evidências devem provar que é possível efetuar as seguintes ações sem o conhecimento ou consentimento do usuário:

\begin{alineas}
	\item scripts provenientes de domínios diferentes podem observar o conteúdo de páginas da web, incluindo identificações, senhas, códigos de cartão de crédito e números de telefone, desde que essas informações estejam presentes na estrutura da página;
	\item scripts agindo em extensões do navegador podem observar o conteúdo de páginas da web e de seus subcomponentes (\textit{iframes});
	\item scripts de qualquer natureza podem registrar o comportamento do usuário ao interagir com a página, capturando eventos de teclado e de mouse;
	\item scripts de qualquer natureza conseguem interceptar funcionalidades do navegador para extrair informações que transitem pelas interfaces de programação do DOM e de Javascript.
\end{alineas}

Artefatos derivados dos casos de testes, codificados como páginas da web, serão insumo das atividades de avaliação do método proposto. O método será considerado eficaz se os problemas evidenciados forem neutralizados através de sua aplicação.


\subsection{Proposição do método}
O objetivo desta atividade é projetar um método de encapsulamento através do qual desenvolvedores de páginas da web possam definir componentes de HTML imunes ao vazamento de informação por meio de Javascript. O método precisará atender aos seguintes requisitos:

\begin{alineas}
	\item Permitir que qualquer combinação de elementos HTML seja encapsulável em um componente inviolável por scripts externos a si;
	\item Ser compatível com bibliotecas e \textit{frameworks} de desenvolvimento em Javascript, HTML e CSS;
	\item Estabelecer um protocolo de confiança entre si e agentes (scripts) externos;
	\item Sob esse mesmo protocolo, expor uma interface de programação para a leitura e modificação das informações encapsuladas;
	\item Ser compatível com recursos padronizados e não-experimentais de HTML e Javascript.
\end{alineas}

O resultado desta atividade é uma especificação técnica da solução proposta, incluindo pré-requisitos e limitações de uso.


\subsection{Implementação-modelo do método proposto}
Partindo da especificação do método, será construída uma implementação-modelo para fins de validação. Esta atividade tem a finalidade de gerar um artefato de HTML e Javascript compatível com os requisitos estabelecidos pela proposta. Como requisito adicional e específico para esta atividade, a implementação-modelo deve ser transparentemente integrável aos artefatos dos casos de teste, de forma a não produzir nenhuma mudança no comportamento apresentado por eles aos usuários.


\subsection{Avaliação do método}
Nesta tarefa, artefatos que materializam os casos de teste serão modificados para que sejam compatíveis com implementação-modelo do método proposto. O método será assim submetido à avaliação de sua eficácia frente às vulnerabilidades e sua compatibilidade em relação aos pré-requisitos do método. Parte desta tarefa será dedicada à automação dos casos de teste na forma de scripts compatíveis com o \textit{framework} Selenium\footnote{http://www.seleniumhq.org/}. Os scripts de automação serão agregados ao conjunto de artefatos de código derivados deste trabalho.


\subsection{Síntese dos resultados}
A avaliação do método produzirá observações objetivas, como a eficácia contra vulnerabilidades e reflexos no desempenho do navegador, bem como impressões subjetivas como a facilidade de aplicação e a severidade das restrições impostas pelo método. A partir dessas observações será elaborada uma síntese dos resultados alcançados, incluindo uma reflexão sobre a abrangência e utilidade dos artefatos produzidos. As realizações e limitações deste trabalho serão  comparadas com as de trabalhos relacionados, posicionando definitivamente estas contribuições dentro do panorama de segurança da informação.