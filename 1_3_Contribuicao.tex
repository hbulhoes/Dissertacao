%!TEX root = Projeto.tex
\section{Resultados esperados e contribuições}

%A SEÇÃO "OBJETIVO" ESTABELECEU QUE O TRABALHO VAI TRATAR DA PROTEÇÃO (OU NÃO) QUE UMA TECNOLOGIA COMO "SHADOW DOM" OFERECE AOS USUÁRIOS DE PÁGINAS WEB. NESTA SEÇÃO: QUE RESULTADOS SÃO ESPERADOS DESSE ESTUDO [INCLUIR: REVISÃO/LANDSCAPE (SIM) DO ESTADO DA ARTE E DAS VULNERABILIDADES DOS NAVEGADORES] [INCLUIR: MAPEAMENTO DOS ASPECTOS QUALITATIVOS DE UMA SOLUÇÃO DE INFOSEC PARA JAVASCRIPT EM NAVEGADOR]? COMO ESSES RESULTADOS SÃO IMPORTANTES PARA O AVANÇO NO ESTADO DA ARTE? COMO OS ARTEFATOS DERIVADOS DO TRABALHO PODEM CONTRIBUIR PARA A SEGURANÇA DA INFORMAÇÃO EM PÁGINAS DA WEB?

Este trabalho contribui com o estado da arte em segurança nas aplicações da web pela investigação do potencial de inviolabilidade da informação oferecido por tecnologias não experimentais, presentes nas APIs dos navegadores, contrastando-as com uma abordagem de referência baseada no controle do fluxo de informação (IFC -- \textit{information flow control}). Esse referencial é relevante pois IFC, que fundamentalmente redefine o trânsito de dados em tempo de execução na linguagem Javascript, ofereceria o nível mais alto de segurança se fosse integrada por padrão às APIs dos navegadores. Sem que esse recurso esteja disponível aos desenvolvedores e usuários, é oportuno, portanto, que seja avaliado o nível de segurança das ferramentas de alcance geral.

%Trata-se de uma preocupação ausente na literatura sobre segurança da informação em Javascript. Nela parece existir uma distância entre as inovações introduzidas pelos navegadores e os tópicos sensíveis à comunidade acadêmica. Trazer as duas vertentes em torno de um objeto de pesquisa -- o método investigado por este trabalho -- parece não apenas possível como relevante por sua aplicabilidade imediata, caso se prove suficientemente eficaz.
