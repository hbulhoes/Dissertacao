%!TEX root = Projeto.tex
\section{Contribuições}

Este trabalho contribui para o estado da arte pela proposição de uma abordagem discricionária para o controle do acesso à informação armazenada no DOM. Esta é uma trilha divergente de trabalhos anteriores que exploram o controle de acesso mandatório como modelo, baseando-se, sobretudo, no controle do fluxo da informação.

Contribuições como ``COWL'' \cite{Stefan2014}, ``JSFlow'' \cite{Hedin2016}, o instrumentador de \textit{bytecode} de \cite{Bichhawat2014}, ``JCShadow'' \cite{Patil2011} ou ``SessionGuard'' \cite{Patil2017} se baseiam em tecnologia experimental, em contexto acadêmico de utilização. Até o presente momento, apenas COWL (\textit{Confinement with Origin Web Labels}) é candidato a transformar-se em recomendação pelo comitê W3C\cite{W3C:COWL}.

Na contramão das propostas experimentais, e inspirado por elas, este trabalho contribui com uma abordagem baseada em APIs já disponíveis em navegadores de ampla utilização, representando uma opção tangível para o desenvolvedor de aplicações web e seus usuários.

% Outra contribuição esperada é um algoritmo para a detecção de sobrecargas em métodos e propriedades do DOM. Esse algoritmo supre a necessidade de se prevenir formas de intrusão onde um \textit{script} intercepta determinadas APIs do navegador para monitorar e, eventualmente, capturar informação que transite através de suas chamadas. Tal intrusão permitiria tomar controle das próprias APIs utilizadas pelo método, abrindo a possibilidade de vazamento de informação e inutilizando a estratégia.
