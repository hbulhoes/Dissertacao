%!TEX root = Projeto.tex
\section{Contribuições}

Este trabalho contribui para o estado da arte pela proposição de uma abordagem discricionária para o controle do acesso à informação armazenada no DOM. Esta é uma trilha divergente de trabalhos anteriores que exploram o controle de acesso mandatório como modelo.

Ainda que, em teoria, um modelo mandatório de controle tenda a exercer um nível de proteção maior que aquele oferecido por modelos discricionários \cite[p. 46]{CNSS2015} \cite[p. 4]{Foster1998}, os navegadores da web comuns não o implementam. Para tanto, precisam ser estendidos por software experimental como COWL \cite{Stefan2014}, JSFlow \cite{Hedin2016}, o instrumentador de \textit{bytecode} de \cite{Bichhawat2014}, ou JCShadow \cite{Patil2011}. Não são trabalhos de alcance generalizado, e até o presente momento apenas COWL é candidato a transformar-se em recomendação pelo comitê W3C \cite{W3C:COWL}.

Ao contrário das propostas experimentais, este trabalho contribui com uma abordagem baseada em APIs já disponíveis em navegadores de ampla utilização, algo tangível para o desenvolvedor de aplicações web e de seus usuários.

% Outra contribuição esperada é um algoritmo para a detecção de sobrecargas em métodos e propriedades do DOM. Esse algoritmo supre a necessidade de se prevenir formas de intrusão onde um \textit{script} intercepta determinadas APIs do navegador para monitorar e, eventualmente, capturar informação que transite através de suas chamadas. Tal intrusão permitiria tomar controle das próprias APIs utilizadas pelo método, abrindo a possibilidade de vazamento de informação e inutilizando a estratégia.
