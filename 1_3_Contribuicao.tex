%!TEX root = Projeto.tex
\section{Contribuições}

\cite{Stefan2014} propõe um navegador cujo ambiente de execução de Javascript é refatorado para suportar controle de acesso ao estilo MAC, em detrimento da abordagem alinhada ao estilo DAC implementado pelos navegadores comuns. MAC, como proposto por \cite{Stefan2014}, é alcançado pela implementação do controle do fluxo da informação (IFC) no \textit{runtime} de Javascript como premissa para a segurança da informação. Outros trabalhos \cite{Hedin2016, Bichhawat2014} propõem abordagens similares, vinculadas ao emprego do IFC.

\cite{Magazinius2014} e \cite{DeRyck2012} comparam estratégias voltadas para a segurança da informação e seus casos de uso. Aplicar IFC, no momento, é uma opção que exclui todos os navegadores comuns, exigindo a utilização de software experimental. Esta não é uma alternativa para o desenvolvedor de aplicações web, já que estas são disponibilizadas para uso em qualquer navegador, em múltiplas plataformas. Em oposição a essas propostas, este trabalho contribui com uma abordagem DAC baseada em APIs disponíveis em navegadores de ampla utilização.

% Outra contribuição esperada é um algoritmo para a detecção de sobrecargas em métodos e propriedades do DOM. Esse algoritmo supre a necessidade de se prevenir formas de intrusão onde um \textit{script} intercepta determinadas APIs do navegador para monitorar e, eventualmente, capturar informação que transite através de suas chamadas. Tal intrusão permitiria tomar controle das próprias APIs utilizadas pelo método, abrindo a possibilidade de vazamento de informação e inutilizando a estratégia.
