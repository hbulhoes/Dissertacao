%!TEX root = Projeto.tex
\section{Resultados esperados e contribuições}

%A SEÇÃO "OBJETIVO" ESTABELECEU QUE O TRABALHO VAI TRATAR DA PROTEÇÃO (OU NÃO) QUE UMA TECNOLOGIA COMO "SHADOW DOM" OFERECE AOS USUÁRIOS DE PÁGINAS WEB. NESTA SEÇÃO: QUE RESULTADOS SÃO ESPERADOS DESSE ESTUDO [INCLUIR: REVISÃO/LANDSCAPE (SIM) DO ESTADO DA ARTE E DAS VULNERABILIDADES DOS NAVEGADORES] [INCLUIR: MAPEAMENTO DOS ASPECTOS QUALITATIVOS DE UMA SOLUÇÃO DE INFOSEC PARA JAVASCRIPT EM NAVEGADOR]? COMO ESSES RESULTADOS SÃO IMPORTANTES PARA O AVANÇO NO ESTADO DA ARTE? COMO OS ARTEFATOS DERIVADOS DO TRABALHO PODEM CONTRIBUIR PARA A SEGURANÇA DA INFORMAÇÃO EM PÁGINAS DA WEB?

Abordagens não-experimentais contra vazamentos de dados são preocupação escassa na literatura sobre segurança da informação em Javascript. Este trabalho, ao descrever uma estratégia em que a inviolabilidade da informação contida no software navegador seja assegurada pelo emprego de APIs padronizadas, não-experimentais, contribui com o estado da arte em segurança nas aplicações da web.

Derivado da estratégia avaliada, será proposto um método para encapsulamento de informação que torne determinadas regiões do DOM invisíveis a \textit{scripts} maliciosos. A eficácia do método será validada por testes que simulem, sob diferentes condições, casos específicos de ataques e tentativas de acesso indevido ao conteúdo das regiões protegidas do DOM. É esperado que o método impeça que scripts não autorizados observem e manipulem o conteúdo dessas regiões, bem como torne indetectáveis os eventos de interação do usuário com os elementos das regiões protegidas.

Além do método para encapsulamento, outra contribuição esperada é um algoritmo para a detecção de sobrecargas inesperadas em métodos e propriedades do DOM. Sem tal algoritmo, o método proposto não terá como prevenir uma forma de intrusão onde um \textit{script} intercepta determinadas APIs do navegador para monitorar e, eventualmente, capturar informação que transite através de suas chamadas. Tal intrusão permitiria tomar controle das próprias APIs utilizadas pelo método, abrindo a possibilidade de vazamento de informação e inutilizando a estratégia.

%Trata-se de uma preocupação ausente na literatura sobre segurança da informação em Javascript. Nela parece existir uma distância entre as inovações introduzidas pelos navegadores e os tópicos sensíveis à comunidade acadêmica. Trazer as duas vertentes em torno de um objeto de pesquisa -- o método investigado por este trabalho -- parece não apenas possível como relevante por sua aplicabilidade imediata, caso se prove suficientemente eficaz.
