%!TEX root = Projeto.tex
\newpage
\begin{resumo}
\normalsize

% Breve contextualização do problema
Os programas navegadores oferecem amplos recursos para a execução de aplicações voltadas para a web. Suas funcionalidades, porém, podem expor a informação dos usuários a riscos de vazamento e adulteração. Ainda que os navegadores venham sendo melhorados para que consigam detectar e mitigar alguns desses riscos, esse esforço é marcado por concessões às capacidades esperadas pelas aplicações web, fazendo com que a segurança da informação no navegador seja um campo de conhecimento com iniciativas e padrões consideradas incoerentes entre si. Por causa disso, desenvolvedores de aplicações nem sempre podem prever o grau de exposição dos dados de seus usuários.
% Dica do que pode ser feito para minimizar o problema
Seria importante que fosse possível determinar, de modo programável e imperativo, que esses dados ficassem fora do alcance de participantes não confiáveis, particularmente \scripts.
% Proposta do trabalho
Para essa finalidade, este trabalho propõe uma abordagem que proporcione ao desenvolvedor uma barreira de proteção incorporada às aplicações. Nesta proposta, o desenvolvedor tem controle direto sobre a exposição das informações que considerar sensíveis, em contraste com o controle indireto, declarativo e tolerante a vazamento de informação empregado atualmente nos navegadores.
% Como será validado
A validação dessa proposta ocorrerá por meio de um protótipo de sistema submetido a situações de risco de vazamento da informação em páginas web, em correspondência com ocorrências documentadas pela literatura. O protótipo será preparado para que a abordagem proposta, sob a forma de um \script baseado em APIs padronizadas de HTML e Javascript, seja colocada à prova no seu propósito de neutralizar tais situações de risco.


\vspace{\onelineskip}

\noindent
\textbf{Palavras-chave:} \imprimirpalavraschave
\end{resumo}

% resumo em inglês
%\begin{resumo}[Abstract]
%%\begin{otherlanguage*}{english}
%Resumo da dissertação em inglês.