%!TEX root = Projeto.tex
\newpage
\begin{resumo}
\normalsize

% Breve contextualização do problema
O desenvolvimento de aplicações destinadas ao ambiente de execução dos navegadores da web requer atenção aos riscos de vazamento da informação de usuário contida no navegador, sejam esses riscos explorados de forma intencional e maliciosa, sejam eles acionados por consequência acidental do próprio funcionamento de uma aplicação web. Embora ao longo do tempo os navegadores venham sendo melhorados para que consigam detectar e mitigar alguns desses riscos, esse esforço é marcado por concessões às funcionalidades esperadas pelas aplicações web, fazendo com que a segurança da informação no navegador seja um campo de conhecimento com práticas consideradas incoerentes entre si. Um cipoal de iniciativas e padrões de segurança se impõe aos desenvolvedores de aplicações, que nem sempre podem prever o grau de exposição dos dados de seus usuários no navegador uma vez que esta propriedade é produto de uma combinação entre a configuração dos servidores de aplicação, o nível de confiança entre as partes componentes das páginas web, o conjunto de extensões ativadas pelo navegador, e o teor dos \textit{scripts} envolvidos.
% Dica do que pode ser feito para minimizar o problema
Recursos de programação poderiam ser empregados para que informações sensíveis ficassem fora do alcance de participantes não confiáveis, particularmente \textit{scripts}.
% Proposta do trabalho
No âmbito da segurança da informação, a proposta deste trabalho é avaliar a viabilidade de uma abordagem que proporcione ao desenvolvedor uma barreira de proteção incorporada às aplicações, complementar aos recursos de segurança amplamente incorporados ao \textit{backend} e ao navegador.
% Como será validado
A validação dessa proposta ocorrerá por meio de um protótipo de sistema submetido a situações de risco de vazamento da informação em páginas web, em correspondência com ocorrências documentadas pela literatura. O protótipo será preparado para que a abordagem proposta, sob a forma de um script baseado em APIs padronizadas de HTML e Javascript, seja colocada à prova no seu propósito de neutralizar tais situações de risco.


\vspace{\onelineskip}

\noindent
\textbf{Palavras-chave:} \imprimirpalavraschave
\end{resumo}

% resumo em inglês
%\begin{resumo}[Abstract]
%%\begin{otherlanguage*}{english}
%Resumo da dissertação em inglês.