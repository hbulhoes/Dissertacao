%!TEX root = Projeto.tex
\newpage
\begin{resumo}
\normalsize

Impulsionados por recursos avançados de HTML e Javascript, os programas navegadores da web tornaram-se plataformas de desenvolvimento e publicação de aplicativos. O aumento em escopo de funcionalidade dos navegadores, porém, vem acompanhado de vulnerabilidades de segurança da informação que podem expor dados do usuário ao uso não autorizado. Muitas dessas vulnerabilidades se manifestam de formas sutis, fora do alcance das políticas de segurança padronizadas e adotadas pela indústria, e afetando a confidencialidade da informação que trafega pelo navegador. Para neutralizar esse problema diversas iniciativas experimentais têm sido propostas, sem que tenham encontrado, até o momento, adoção em massa. Ao mesmo tempo, novos recursos de programação têm sido incorporados aos navegadores, e alguns deles, quando utilizados em conjunto, levantam a possibilidade de implementação do encapsulamento da informação, tornando-a invisível a agentes não autorizados. O objetivo deste trabalho é propor um método que, fundamentado em capacidades de programação padronizadas e difundidas, implemente o encapsulamento da informação em HTML e Javascript, o que poderia tornar realidade alguns dos requisitos de segurança que apenas abordagens experimentais têm conseguido assegurar. Para tanto, este trabalho apresentará uma especificação desse método, uma implementação modelo e uma avaliação da eficácia do método em relação a determinados casos de teste que evidenciam problemas de confidencialidade da informação.

\vspace{\onelineskip}

\noindent
\textbf{Palavras-chave:} \imprimirpalavraschave
\end{resumo}

% resumo em inglês
%\begin{resumo}[Abstract]
%%\begin{otherlanguage*}{english}
%Resumo da dissertação em inglês.